\documentclass[aspectratio=169]{beamer}

\usetheme{default}
\setbeamertemplate{navigation symbols}{}
\setbeamertemplate{enumerate item}{\color{navy}\arabic{enumi}.}
\setbeamertemplate{itemize item}{\color{black}\textbullet}
\setbeamertemplate{itemize subitem}{\color{black}\textbullet}
\usepackage{booktabs}
\usepackage{xcolor}
\usepackage{tikz}
\usetikzlibrary{shapes,arrows,positioning}
\definecolor{navy}{RGB}{0, 0, 128}
\definecolor{lightblue}{RGB}{230,240,250}
\definecolor{darkgreen}{RGB}{0,100,0}
\definecolor{lightgreen}{RGB}{230,250,230}
\newcommand{\highlight}[1]{\colorbox{lightblue}{$\displaystyle\textcolor{navy}{#1}$}}
\newcommand{\highlighttext}[1]{\colorbox{lightblue}{\textcolor{navy}{#1}}}
\newcommand{\highlightgreen}[1]{\colorbox{lightgreen}{$\displaystyle\textcolor{darkgreen}{#1}$}}

\usepackage{hyperref}
\hypersetup{
    colorlinks=true,
    linkcolor=navy,
    urlcolor=navy,
    citecolor=navy
}

\begin{document}


\begin{frame}

\textcolor{navy}{Central Limit Theorem}

\bigskip{}

\onslide<2->{
For \textit{iid} random variables $X_1, \ldots, X_N$ with mean $\mu$ and variance $\sigma^2$:
$$\frac{\sqrt{N}(\overline{X}_N - \mu)}{\sigma} \xrightarrow{d} N(0,1)$$
}
\onslide<3->{
Remarkable: works for \textit{any} underlying distribution (with finite variance)
\bigskip\par
}
\onslide<4->{
\textit{Economics applications:}
\bigskip\par
\begin{itemize}
\itemsep1em
\item<4-> Provides asymptotic sampling distribution to be used in hypothesis testing
\item<5-> $X_1,\ldots,X_N$ can be nearly any statistics from repeated samples of the population
\item<6-> e.g. medians, OLS regression coefficients, MLE parameter estimates, ...
\end{itemize}
}

\end{frame}



\begin{frame}

% Thanks to this source for elucidating some of these concepts: https://borisburkov.net/2023-04-30-1/

\textcolor{navy}{Extreme Value Theory}
\bigskip

\onslide<2->{
What if we want the limiting distribution of $\max_i \left\{X_i\right\}_{i=1}^N$ instead of $\overline{X}=\frac{1}{N}\sum_i X_i$?
}

\bigskip{}

\onslide<3->{
Would that distribution be Normal? Is there an equivalent ``Central Limit Theorem''?
}

\bigskip{}

\onslide<4->{
\textit{Economics applications:}
\bigskip\par
\begin{itemize}
\itemsep1em
\item<5-> Discrete choice: agents maximize utility across alternatives
\item<6-> Auctions: winner determined by highest valuation
\item<7-> Market structure: largest firm, highest productivity when winner takes all
\end{itemize}
}

\end{frame}






\begin{frame}

\textcolor{navy}{Fisher-Tippett-Gnedenko Theorem}

\bigskip{}

The maximum of $N$ \textit{iid} random variables converges to \textcolor{navy}{only three types} of distributions:

\bigskip{}

\begin{itemize}
\itemsep1.5em
\item<2-> \textit{Type I (Gumbel):} $F(x) = e^{-e^{-\frac{x-\mu}{\beta}}}$ with $x \in \mathbb{R}$
\item<3-> \textit{Type II (Fr\'{e}chet):} $F(x) = e^{-\left(\frac{x-\mu}{\beta}\right)^{-\alpha}}$ with $x > \mu$
\item<4-> \textit{Type III (Weibull):} $F(x) = e^{-\left(\frac{\mu-x}{\beta}\right)^{\alpha}}$ with $x < \mu$
\end{itemize}

\bigskip{}

\onslide<5->{
But the limiting distribution may not exist! Depends on distribution underlying $X_i$'s
}

\end{frame}

\begin{frame}

\textcolor{navy}{Domain of Attraction}

\bigskip{}

Which parent distributions lead to which extreme value type?

\bigskip{}

\begin{itemize}
\itemsep1.5em
\item<2-> \textit{Gumbel:} Exponential-type tails (normal, exponential, gamma, lognormal)
\item<3-> \textit{Fr\'{e}chet:} Heavy tails (Pareto, Cauchy, Student-$t$)
\item<4-> \textit{Weibull:} Bounded support with finite upper end point (uniform, beta)
\end{itemize}

\bigskip{}

\onslide<5->{
Key condition: tail behavior of $F(x)$ as $x \to \infty$
}

\end{frame}



\begin{frame}

\onslide<1->{
In \textcolor{navy}{extreme value theory}, we study the distribution of the \textit{maximum}:
$$ M_N = \max\{X_1, \ldots, X_N\} $$
}
\onslide<2->{
\par In \textcolor{navy}{discrete choice}, we model an agent choosing the alternative with the \textit{highest utility}:
$$ j^* = \arg\max_j \, U_{ij} = \arg\max_j \, (u_{ij} + \epsilon_{ij}) $$
}
\onslide<3->{
\par Both involve a \textit{maximum of random variables}, so it's useful to assume $\epsilon$'s come from an extreme value distribution family
}
\bigskip{}

\onslide<4->{
The T1EV assumption also gives us closed-form choice probabilities and $\mathbb{E}(U)$
}

\end{frame}





\begin{frame}

$J$ alternatives, $\epsilon_{ij}\overset{iid}{\sim}$ Type I extreme value; $U_{ij} = u_{ij} + \epsilon_{ij}$; $F(\epsilon) = e^{-e^{-\epsilon}}$ (CDF)

\bigskip{}

\onslide<2->{
We want to know the formula for the probability of choosing option $j$:

\begin{align*}
P_{ij} &= \text{Pr}(u_{ij} + \epsilon_{ij} > u_{ik} + \epsilon_{ik}\,\,\,\forall\,\,\,k\neq j)\\[1em]
&= \frac{\exp\left(u_{ij}\right)}{\sum_k \exp\left(u_{ik}\right)}
\end{align*}
}

\bigskip{}

\onslide<3->{
We also want to know the formula for expected utility:

\begin{align*}
\mathbb{E}_\epsilon \max_k \left\{u_{ik}+\epsilon_{ik}\right\} &= \log\left(\sum_k \exp\left(u_{ik}\right)\right) + \gamma
\end{align*}
}

\end{frame}




\begin{frame}

\textcolor{navy}{Central Limit Theorem vs Extreme Value Theory}

\bigskip{}

\begin{center}
\begin{tabular}{lcc}
\toprule
& \textbf{CLT} & \textbf{EVT} \\
\midrule
\onslide<2->{Object of interest & $\displaystyle \frac{1}{N}\sum_{i=1}^N X_i$ & $\displaystyle \max_i \left\{X_i\right\}_{i=1}^{N}$} \\[0.6em]
\onslide<3->{Normalization & $\displaystyle \frac{\sqrt{N}(\overline{X}_N - \mu)}{\sigma}$ & $\displaystyle \frac{M_N - b_N}{a_N}$} \\[0.6em]
\onslide<4->{Limiting distribution & Normal & Gumbel / Fr\'{e}chet / Weibull} \\[0.6em]
\onslide<5->{Number of limit types & One & Three} \\[0.6em]
\onslide<6->{Depends on & Finite variance & Tail behavior of $F(x)$} \\[0.6em]
\onslide<7->{Economic applications & Sampling distributions, & Choice models,} \\
\onslide<7->{ & inference & auctions, extremes} \\
\bottomrule
\end{tabular}
\end{center}


\end{frame}

\end{document}
