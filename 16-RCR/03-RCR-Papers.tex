\documentclass[aspectratio=169]{beamer}

\usetheme{default}
\setbeamertemplate{navigation symbols}{}
\setbeamertemplate{enumerate item}{\color{navy}\arabic{enumi}.}
\setbeamertemplate{itemize item}{\color{black}\textbullet}
\setbeamertemplate{itemize subitem}{\color{black}\textbullet}
\usepackage{booktabs}
\usepackage{xcolor}
\usepackage{tikz}
\usetikzlibrary{shapes,arrows,positioning}
\definecolor{navy}{RGB}{0, 0, 128}
\definecolor{lightblue}{RGB}{230,240,250}
\definecolor{darkgreen}{RGB}{0,100,0}
\definecolor{lightgreen}{RGB}{230,250,230}
\newcommand{\highlight}[1]{\colorbox{lightblue}{$\displaystyle\textcolor{navy}{#1}$}}
\newcommand{\highlighttext}[1]{\colorbox{lightblue}{\textcolor{navy}{#1}}}
\newcommand{\highlightgreen}[1]{\colorbox{lightgreen}{$\displaystyle\textcolor{darkgreen}{#1}$}}

\usepackage{hyperref}
\hypersetup{
    colorlinks=true,
    linkcolor=navy,
    urlcolor=navy,
    citecolor=navy
}

\begin{document}

\begin{frame}
\centering
\includegraphics[width=0.7\textwidth]{AET_2005_JPE_cover.png}
\end{frame}


\begin{frame}

\begin{itemize}
\itemsep1.5em
\item<1-> Pioneer the method of bounding $Corr(d,\varepsilon)$ by looking at $Corr(d,X\beta)$

\item<2-> Setting: estimating causal effect of attending Catholic high school

\item<3-> Lots of selection, no random variation available

\item<4-> Argue that $Corr(d,\varepsilon)=Corr(d,X\beta)$ is a good bound
\end{itemize}

\end{frame}

\begin{frame}

Why assume $Corr(d,\varepsilon)=Corr(d,X\beta)$?

\bigskip

\begin{itemize}
\itemsep1.5em
\item<2-> $X$ is only a subset of everything that affects $y$

\item<3-> Data collected for multiple purposes

\item<4-> Data costly to collect, some variables impossible to measure

\item<5-> Thus $X$ is probably a \textcolor{navy}{random} subset of everything affecting $y$
\end{itemize}

\end{frame}

\begin{frame}

$Corr(d,\varepsilon)\overset{?}{\lesseqqgtr}Corr(d,X\beta)$

\bigskip

\begin{itemize}
\itemsep1.5em
\item<2-> Application-specific question requiring careful thinking

\item<3-> What are sources of selection bias? Why is $R^2$ low?

\item<4-> Selection bias? Measurement error? Irreducible uncertainty?

\item<5-> If lots of irreducible uncertainty: $Corr(d,\varepsilon)\ll Corr(d,X\beta)$

\item<6-> Opposite true if not much irreducible uncertainty

\item<7-> Typically assume $Corr(d,\varepsilon)$ and $Corr(d,X\beta)$ have same sign
\end{itemize}

\end{frame}


\begin{frame}
\centering
\includegraphics[width=0.52\textwidth]{Krauth_2016_JEM_cover.png}
\end{frame}



\begin{frame}

\begin{itemize}
\itemsep1.5em
\item<1-> Generalizes Altonji, Elder and Taber (2005)

\item<2-> Allows for \textcolor{navy}{relative correlation restriction (RCR)}:
\begin{align*}
Corr(d, \varepsilon) = \lambda Corr(d, X\beta)
\end{align*}
\end{itemize}

\end{frame}

\begin{frame}

Two uses of $\lambda$:

\bigskip

\begin{enumerate}
\itemsep1.5em
\item<2-> Assume $\lambda \in [\lambda_L, \lambda_H]$ and estimate corresponding $\alpha$'s in interval $[\alpha_L, \alpha_H]$

\item<3-> Estimate $\alpha$ by OLS, then find smallest (absolute) value of $\lambda$ such that OLS estimate is statistically zero
\end{enumerate}

\end{frame}


\begin{frame}
\centering
\includegraphics[width=0.52\textwidth]{Oster_2019_JBES_cover.png}
\end{frame}



\begin{frame}

\begin{itemize}
\itemsep1.5em
\item<1-> Also generalizes Altonji et al. (2005)

\item<2-> Focuses on comparing movements in $\alpha$ with movements in $R^2$

\item<3-> Intuition: if we could observe all unobservables, then $R^2=1$

\item<4-> Value of $\alpha$ when $R^2=1$ represents true causal value

\item<5-> If measurement error in $y$, consider $R_{\max}<1$ instead

\item<6-> Implementation closely similar to Krauth (2016)
\end{itemize}

\end{frame}

\end{document}