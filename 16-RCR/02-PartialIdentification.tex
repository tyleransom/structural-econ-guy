\documentclass[aspectratio=169]{beamer}

\usetheme{default}
\setbeamertemplate{navigation symbols}{}
\setbeamertemplate{enumerate item}{\color{navy}\arabic{enumi}.}
\setbeamertemplate{itemize item}{\color{black}\textbullet}
\setbeamertemplate{itemize subitem}{\color{black}\textbullet}
\usepackage{booktabs}
\usepackage{xcolor}
\usepackage{tikz}
\usetikzlibrary{shapes,arrows,positioning}
\definecolor{navy}{RGB}{0, 0, 128}
\definecolor{lightblue}{RGB}{230,240,250}
\definecolor{darkgreen}{RGB}{0,100,0}
\definecolor{lightgreen}{RGB}{230,250,230}
\newcommand{\highlight}[1]{\colorbox{lightblue}{$\displaystyle\textcolor{navy}{#1}$}}
\newcommand{\highlighttext}[1]{\colorbox{lightblue}{\textcolor{navy}{#1}}}
\newcommand{\highlightgreen}[1]{\colorbox{lightgreen}{$\displaystyle\textcolor{darkgreen}{#1}$}}

\usepackage{hyperref}
\hypersetup{
    colorlinks=true,
    linkcolor=navy,
    urlcolor=navy,
    citecolor=navy
}

\begin{document}

\begin{frame}
Let's consider a basic linear model with cross-sectional data:
\onslide<2->{
\begin{align}
    y &= \alpha d + X\beta + \varepsilon
\end{align}
where $d$ is a treatment indicator; $X$'s are other observable covariates
}
\bigskip\par
\onslide<3->{
Assume that $Corr(X,\varepsilon)=0$ but that $Corr(d,\varepsilon)\neq 0$
}
\end{frame}

\begin{frame}
\bigskip\par
\onslide<1->{
How can $Corr(d,\varepsilon)\neq 0$? Rewrite (1):
}
\onslide<2->{
\begin{align}
    y &= \alpha d + X\beta + \underbrace{W\gamma + \nu}_{\varepsilon}
\end{align}
with $Corr(d,W)\neq 0$ but $Corr(X,W) = 0$; $\nu$ is purely idiosyncratic
}
\bigskip\par
\onslide<3->{
$W$'s are variables that are unobservable or excluded from the model
}
\end{frame}


\begin{frame}

Some terminology:
\bigskip

\begin{itemize}
\itemsep1.5em
\item<2-> $Corr(d,X\beta)$ is called \textcolor{navy}{selection on observables}

\item<3-> $Corr(d,\varepsilon)$ is called \textcolor{navy}{selection on unobservables}

\item<4-> Selection on unobservables is what confounds our original regression model

\item<5-> Can always measure $Corr(d,X\beta)$ in the data; can never measure $Corr(d,\varepsilon)$
\end{itemize}
\bigskip\par

\onslide<6->{
Can we use selection on observables to tell us amount of selection on unobservables?
}



\end{frame}


\begin{frame}

\begin{align}
    y &= \alpha d + X\beta + \underbrace{W\gamma + \nu}_{\varepsilon} \tag{2}
\end{align}

Key question: Can we use $Corr(d,X\beta)$ to learn about the \textcolor{navy}{causal effect} of $d$ on $y$?
\bigskip

\begin{itemize}
\itemsep1.5em
\item<2-> One extreme: Assume unconfoundedness, i.e. $Corr(d,\varepsilon)=0$

\item<3-> Another extreme: Assume $Corr(d,\varepsilon)=Corr(d,X\beta)$, e.g. if  $X{\underset{\text{\tiny random}}{\subset}} W$

\item<4-> Combine $Corr(d,X\beta)$ with assumption about $Corr(d,\varepsilon)$ to properly ``shade'' $\hat{\alpha}$
\end{itemize}

\end{frame}


\begin{frame}
We can follow these steps to obtain a plausibly causal $\hat{\alpha}$:
\bigskip\par
\begin{enumerate}
\itemsep1.5em
\item<2-> Regress $y$ on $d$
\item<3-> Regress $y$ on $d$ and $X$
\item<4-> Observe how $\hat{\alpha}$ and $R^2$ change across steps 1 and 2
\item<5-> Make an assumption on $\lambda$ such that $Corr(d,\varepsilon) = \lambda Corr(d,X\beta)$
\item<6-> Use all of the above information to get an \textcolor{navy}{interval} on plausible $\hat{\alpha}$ TE
\end{enumerate}
\end{frame}


\end{document}