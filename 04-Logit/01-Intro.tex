\documentclass[aspectratio=169]{beamer}

\usetheme{default}
\setbeamertemplate{navigation symbols}{}
\setbeamertemplate{itemize item}{\color{black}\textbullet}
\setbeamertemplate{itemize subitem}{\color{black}\textbullet}

\begin{document}

\begin{frame}

We model your utility from restaurant $j$ as:

\begin{align}
    U_{ij} &= u_{ij} + \epsilon_{ij}
\end{align}
where:
\begin{itemize}
\item[]<2->
\item<2-> $u_{ij}$ captures \textbf{observable factors}: price, distance, cuisine type, Yelp rating
\item[]<2->
\item<3-> $\epsilon_{ij}$ captures \textbf{unobservable factors}: mood, cravings, things we can't measure
\item[]<3->
\item<4-> Perhaps $u_{\text{Thai}} = -2 \times price + 3 \times Yelp\_rating - 0.5 \times travel\_time$
\item[]<4->
\item<5-> But $\epsilon_{\text{Thai}}$ might be positive tonight because you're craving spice
\end{itemize}

\end{frame}


\begin{frame}

You choose restaurant $j$ if and only if:

\begin{align}
   \onslide<1->{U_{ij} > U_{ik} &\text{ for all } k \neq j \\}
   \onslide<2->{&\Updownarrow\nonumber\\}
   \onslide<2->{u_{ij} + \epsilon_{ij} > u_{ik} + \epsilon_{ik} &\text{ for all } k \neq j}
\end{align}

\onslide<3->{\bigskip{}

Since we (data analysts) can't observe your $\epsilon$'s, we have to think probabilistically:
\begin{align}
P_{ij} &= \text{Pr}(U_{ij} > U_{ik} \text{ for all } k \neq j)
\end{align}}

\onslide<4->{\bigskip{}

How many times out of 100 would someone choose Thai given identical observables?}

\end{frame}



\begin{frame}

Three key properties of choice sets:

\begin{itemize}
\item[]
\item \textbf{Finite}: You can't choose from an infinite number of restaurants (i.e., ``discrete'')
\item[]
\item \textbf{Mutually exclusive}: You can eat at only one restaurant tonight
\item[]
\item \textbf{Exhaustive}: These are the complete set of realistic options
\end{itemize}

\end{frame}

\end{document}