\documentclass[aspectratio=169]{beamer}

\usetheme{default}
\setbeamertemplate{navigation symbols}{}
\setbeamertemplate{itemize item}{\color{black}\textbullet}
\setbeamertemplate{itemize subitem}{\color{black}\textbullet}
\usepackage{xcolor}
\definecolor{navy}{RGB}{0, 0, 128}

\begin{document}

\begin{frame}

Goodness-of-fit typically measured by likelihood ratio index

\bigskip{}

Also commonly called \textbf{McFadden's Pseudo $R^2$}

\bigskip{}

$$\rho = 1 - \frac{LL(\hat{\beta})}{LL(0)}$$

\bigskip{}

\onslide<2->{
\bigskip{}
Range: $\rho \in [0, 1]$
\begin{itemize}
\item[]
\item $\rho = 0$: estimated model no better than no model
\item[]
\item $\rho = 1$: perfect prediction of all choices
\end{itemize}
}

\end{frame}

\begin{frame}

Unlike regression $R^2$, $\rho$ has no intuitive interpretation between 0 and 1 (i.e. ``pseudo'')

\bigskip{}

\onslide<2->{
$\rho$ represents the percentage increase in log-likelihood above zero parameters:

$$\rho = \frac{LL(0) - LL(\hat{\beta})}{LL(0)}$$
}

\bigskip{}

\onslide<3->{
Valid comparisons require:
\begin{itemize}
\item Same dataset
\item Same choice alternatives
\item Same $LL(0)$ baseline
\end{itemize}
}

\bigskip{}

\onslide<4->{
$\uparrow\rho\Rightarrow$ better fit, but unclear meaning of specific values (0.2--0.4 is ``excellent fit'')
}

\end{frame}


\begin{frame}

Another common goodness-of-fit metric is the percent correctly predicted (``accuracy'')

\bigskip{}

This statistic predicts the alternative with highest probability for each individual

\bigskip{}

\onslide<2->{
\textbf{Key Problems:}
\begin{itemize}
\item[]
\item Contradicts the meaning of choice probabilities
\item[]
\item Depends on arbitrary threshold choice (typically 0.5)
\item[]
\item Performs poorly with imbalanced choice sets
\end{itemize}
}

\bigskip{}

\onslide<3->{
Choice probabilities represent expected shares across many repetitions, not individual predictions

\bigskip{}

This approach gives inaccurate market shares and implies perfect information
}

\end{frame}




\end{document}