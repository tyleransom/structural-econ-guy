\documentclass[aspectratio=169]{beamer}

\usetheme{default}
\setbeamertemplate{navigation symbols}{}
\setbeamertemplate{itemize item}{\color{black}\textbullet}
\setbeamertemplate{itemize subitem}{\color{black}\textbullet}
\usepackage{booktabs}

\begin{document}

\begin{frame}

Utility maximization implies

\begin{align}
   u_{ij} + \epsilon_{ij} > u_{ik} + \epsilon_{ik} &\text{ for all } k \neq j
\end{align}

\bigskip{}
\onslide<2->{
With the $\epsilon$'s unobserved, the probability of $i$ making choice $j$ is given by:

\begin{align}
P_{ij} &= \text{Pr}(u_{ij} + \epsilon_{ij} > u_{ij'} + \epsilon_{ij'} \; \forall \; j' \neq j) \nonumber\\
&\nonumber\\
&= \text{Pr}(\epsilon_{ij'} - \epsilon_{ij} < u_{ij} - u_{ij'} \; \forall \; j' \neq j) \label{eq:probs}\\
&\nonumber\\
&= \int_{\epsilon} I(\epsilon_{ij'} - \epsilon_{ij} < u_{ij} - u_{ij'} \; \forall \; j' \neq j) f(\epsilon) d\epsilon\nonumber
\end{align}
}
\end{frame}



\begin{frame}
Regardless of the distribution of the $\epsilon$'s, probabilities don't change when we:

\begin{itemize}
\item[]
\item Add a constant to the utility of all options
\item[]\qquad\qquad or
\item Multiply by a positive number
\end{itemize}

\bigskip{}
\onslide<2->{Thus, we need to make some normalizations to identify preferences:

\begin{itemize}
\item[]
\item Only \textit{differences} in utility matter (location normalization)
\item[]
\item We need to scale the variance of the $\epsilon$'s (scale normalization)
\end{itemize}
}
\end{frame}



\begin{frame}

\onslide<1->{
Suppose we have two options with the following observable utilities:
\begin{align*}
u_{i1} &= \alpha Tall_i + \beta_1 X_i + \gamma Z_1 \\
u_{i2} &= \alpha Tall_i + \beta_2 X_i + \gamma Z_2
\end{align*}
}

\onslide<2->{
Since only differences in utility matter:
\begin{align*}
u_{i1} - u_{i2} &= (\beta_1 - \beta_2) X_i + \gamma (Z_1 - Z_2)
\end{align*}
}

\onslide<3->{
\begin{itemize}
\item[]
\item We can't tell whether tall people are happier (overall) than short people, but we can tell whether they more strongly prefer one option
\item[]
\item We can only obtain \textit{differenced} coefficient estimates on $X$'s
\item[]
\item We can only obtain an estimate of a coefficient that is constant across choice alternatives if its corresponding variable varies by alternative
\end{itemize}
}

\end{frame}



\begin{frame}
Because only differences in utility matter, we end up having one fewer dimension of $\epsilon$

\bigskip{}
\bigskip{}

Rewriting the last line of \eqref{eq:probs} as a $J-1$ dimensional integral over the \textit{differenced} $\epsilon$'s:

\begin{align}
P_{ij} &= \int_{\tilde{\epsilon}} I(\tilde{\epsilon}_{ij'} < \tilde{u}_{ij'} \; \forall \; j' \neq j) g(\tilde{\epsilon}) d\tilde{\epsilon} \tag{4}
\end{align}
where
$\tilde{\epsilon}_j \equiv \epsilon_j - \epsilon_J$, etc. ($J$ is reference alternative)
\bigskip{}



\end{frame}



\begin{frame}

\begin{itemize}
\item[]
\item The need to normalize scale means we can never estimate the variance of $G(\tilde{\epsilon})$
\item[]
\item This contrasts with linear regression models, where we can easily estimate MSE
\item[]
\item<2-> The scale normalization means our $\beta$'s and $\gamma$'s are implicitly divided by an unknown variance term:
\end{itemize}
\vspace{-0.4cm}

\onslide<3->{
\begin{align*}
u_{i1} - u_{i2} &= (\beta_1 - \beta_2) X_i + \gamma (Z_1 - Z_2) \\
&= \tilde{\beta} X_i + \gamma \tilde{Z} \\
&= \frac{\beta^*}{\sigma} X_i + \frac{\gamma^*}{\sigma} \tilde{Z}
\end{align*}

$\tilde{\beta}$ is what we estimate, but we will never know $\beta^*$ because utility is scale-invariant
}
\end{frame}



\begin{frame}

\renewcommand{\arraystretch}{0.8}
\begin{table}
\centering
\begin{tabular}{lccc}
\toprule
                       & (1)    & (2)    & (3) \\
\midrule
Price (\$)              & -0.145 & -0.112 & -0.098 \\
                       & (0.023) & (0.028) & (0.031) \\
Distance (miles)        &        & -0.067 & -0.054 \\
                       &        & (0.015) & (0.018) \\
Weekend                 &        &        & 0.423 \\
                       &        &        & (0.089) \\
\midrule
$N$                     & 2,456  & 2,456  & 2,456 \\
Pseudo $R^2$            & 0.142  & 0.198  & 0.234 \\
\bottomrule
\end{tabular}
\end{table}

\bigskip{}

\textbf{Important:} We cannot directly compare coefficient magnitudes across specifications due to the scale normalization. Each model implicitly rescales coefficients by $1/\sigma$, where $\sigma$ varies with the set of included variables. ($\sigma \downarrow$ as more covariates are included)

\bigskip{}

Instead, use marginal effects for comparison across specifications

\end{frame}



\end{document}