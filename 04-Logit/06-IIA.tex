\documentclass[aspectratio=169]{beamer}

\usetheme{default}
\setbeamertemplate{navigation symbols}{}
\setbeamertemplate{itemize item}{\color{black}\textbullet}
\setbeamertemplate{itemize subitem}{\color{black}\textbullet}
\usepackage{xcolor}
\definecolor{navy}{RGB}{0, 0, 128}

\begin{document}

\begin{frame}

Independence of Irrelevant Alternatives (IIA)

\bigskip{}

One property of multinomial logit: $P_{ij}/P_{ik}$ does not depend on other alternatives

\bigskip{}

\onslide<2->{
\begin{align*}
\frac{P_{ij}}{P_{ik}} &= \frac{e^{u_{ij}}/\sum_{m}e^{u_{im}}}{e^{u_{ik}}/\sum_{m}e^{u_{im}}}\\
&\\
&= \frac{e^{u_{ij}}}{e^{u_{ik}}}\\
&\\
&= e^{u_{ij}-u_{ik}}
\end{align*}
}

\onslide<3->{
\textbf{Economic meaning:} All alternatives are equally substitutable
}

\end{frame}

\begin{frame}

Advantage of IIA: Enables choice-based sampling

\onslide<1->{
\begin{align*}
P_{ij} = \frac{\exp(u_{ij})}{\sum_{k=1}^J\exp(u_{ik})}
\end{align*}
}

\onslide<2->{
Can estimate using conditional likelihood $P_i(j|j\in \{1,\ldots,K\})$ where $K < J$
}

\onslide<3->{
\begin{align*}
\ell(\beta,\gamma|d_i\in \{1,\ldots,K\}) &= \sum_{i=1}^N\left[\sum_{j=1}^{K-1}(d_{ij}=1)(u_{ij}-u_{iK})\right]\\
&\quad - \log\left(1+\sum_{k=1}^K\exp(u_{ik}-u_{iK})\right)
\end{align*}
}

\onslide<4->{
\bigskip{}
\textbf{Practical use:} Oversample rare choices without bias
}

\end{frame}

\begin{frame}

Disadvantage: Proportional substitution problem

\bigskip{}

\onslide<1->{
Suppose there are two ways for someone to go to work: Car or Blue Bus
}

\bigskip{}

\onslide<2->{
Initial shares: Car 50\%, Blue bus 50\%
}

\bigskip{}

\onslide<3->{
Now let's add an identical red bus ... what would you intuit the new shares to be?
}

\onslide<4->{
\bigskip{}

\textbf{Logit prediction:} Car 33\%, Blue bus 33\%, Red bus 33\%

\bigskip{}
}

\onslide<5->{
\textbf{Reality:} Car 50\%, Blue bus 25\%, Red bus 25\%

\bigskip{}
}

\onslide<6->{
\textbf{Problem:} IIA assumes car and buses equally substitutable
}

\end{frame}

\begin{frame}

Economic significance

\bigskip{}


Suppose people drive 1 of 3 types of cars and there's a subsidy for electric cars:
\only<1>{
\begin{itemize}
\item Large gas cars: 66\% $\rightarrow$ ?
\item Small gas cars: 33\% $\rightarrow$ ?
\item Electric cars: 1\% $\rightarrow$ 10\%
\end{itemize}
}

\onslide<2->{
\begin{itemize}
\item Large gas cars: 66\% $\rightarrow$ 60\%
\item Small gas cars: 33\% $\rightarrow$ 30\%  
\item Electric cars: 1\% $\rightarrow$ 10\%
\end{itemize}
}

\onslide<2->{
\bigskip{}

\textbf{Logit assumption:} Large and small gas cars lose share proportionally
}

\onslide<3->{
\bigskip{}

\textbf{Reality:} Electric cars compete more with small gas cars
}

\onslide<4->{
\bigskip{}

\textbf{Policy implication:} Overestimates gas savings from subsidy
}

\end{frame}

\end{document}