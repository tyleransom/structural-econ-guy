\documentclass[aspectratio=169]{beamer}

\usetheme{default}
\setbeamertemplate{navigation symbols}{}
\setbeamertemplate{enumerate item}{\color{navy}\arabic{enumi}.}
\setbeamertemplate{itemize item}{\color{black}\textbullet}
\setbeamertemplate{itemize subitem}{\color{black}\textbullet}
\usepackage{booktabs}
\usepackage{xcolor}
\usepackage{tikz}
\usetikzlibrary{shapes,arrows,positioning}
\definecolor{navy}{RGB}{0, 0, 128}
\definecolor{lightblue}{RGB}{230,240,250}
\definecolor{darkgreen}{RGB}{0,100,0}
\definecolor{lightgreen}{RGB}{230,250,230}
\newcommand{\highlight}[1]{\colorbox{lightblue}{$\displaystyle\textcolor{navy}{#1}$}}
\newcommand{\highlighttext}[1]{\colorbox{lightblue}{\textcolor{navy}{#1}}}
\newcommand{\highlightgreen}[1]{\colorbox{lightgreen}{$\displaystyle\textcolor{darkgreen}{#1}$}}

\usepackage{hyperref}
\hypersetup{
    colorlinks=true,
    linkcolor=navy,
    urlcolor=navy,
    citecolor=navy
}

\begin{document}

\begin{frame}
\centering
\includegraphics[width=0.48\textwidth]{Athey_Imbens_cover.png}
\end{frame}

\begin{frame}

ML can also help us with treatment effect heterogeneity

\bigskip{}

\onslide<2->{
Use regression trees to partition units into groups with similar treatment effects

\bigskip{}

Estimation is ``honest'' because it uses sample splitting:
\bigskip{}
}

\begin{itemize}
\itemsep1.5em
\item<3-> Split the sample in half
\item<4-> Use one subsample to do the partitioning
\item<5-> Use the other subsample to estimate the treatment effects
\end{itemize}

\end{frame}


\begin{frame}
\centering
\includegraphics[width=0.5\textwidth]{Athey_al_2021_JASA_cover.png}
\end{frame}


\begin{frame}

Causal inference is fundamentally a missing data problem

\bigskip{}

\onslide<2->{
This is because we only ever observe $Y = D_0 Y_0 + D_1 Y_1$
}

\bigskip{}

\onslide<3->{
Athey et al. (2021) propose \textcolor{navy}{matrix completion} methods for panel data
\bigskip
}

\begin{itemize}
\itemsep1.5em
\item<4-> This is a credible data imputation technique
\item<5-> Estimate the ATT by imputing $Y_0$ for treated units
\item<6-> Can be extended to take into account within-unit serial correlation
\end{itemize}

\end{frame}

\end{document}