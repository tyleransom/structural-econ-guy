\documentclass[aspectratio=169]{beamer}

\usetheme{default}
\setbeamertemplate{navigation symbols}{}
\setbeamertemplate{itemize item}{\color{black}\textbullet}
\setbeamertemplate{itemize subitem}{\color{black}\textbullet}

\begin{document}


\begin{frame}
Mincer earnings function

\begin{align}
\log(w_{i}) =& \beta_0 + \beta_1 s_{i} + \beta_2 x_{i} + \beta_3 x^2_{i} + \varepsilon_{i}
\label{eq:basicmincer}
\end{align}

\begin{itemize}
    \item<2-> $i$ indexes people
    \item[]<2->
    \item<2-> $s$ denotes years/grades of completed schooling
    \item[]<2->
    \item<2-> $x$ denotes potential work experience; $x=\text{age}-s-6$
    \item[]<2->
    \item<2-> $\beta_1$ is the return to schooling; $100\cdot\beta_1\approx\frac{\%\Delta w}{\Delta s}$
\end{itemize}
\end{frame}


\begin{frame}
We want to add some structure to this equation, but how?

\begin{align*}
\log(w_{i}) =& \beta_0 + \beta_1 s_{i} + \beta_2 x_{i} + \beta_3 x^2_{i} + \varepsilon_{i}
\end{align*}

\begin{itemize}
    \item<2-> We need to dig into $\varepsilon$ since we need $\mathbb{E}[\varepsilon|s,x,x^2]=0$ for $\beta_1$ to be a ``return''
    \item[]<2->
    \item<2-> $\varepsilon$ has at least two problematic components:
    \item[]<2->
\end{itemize}

\begin{enumerate}
    \item<3-> \textbf{Unobservable personal characteristics correlated with $s$ and $w$} (abilities, comparative advantage, family background, ...)
    \item[]<3->
    \item<3-> \textbf{Downstream choices correlated with $s$ and $w$} (occupation, industry, location, ...)
\end{enumerate}

\end{frame}


\begin{frame}\frametitle{}

Since schooling has an up-front cost and long-term benefit, need a dynamic model

\begin{itemize}
    \item[]
    \item[]
    \item period 1: decide how much schooling to get
    \item[]
    \item period 2: choose whether or not to work; if working, receive $\ln w$ by equation \eqref{eq:basicmincer}
    \item[]
    \item individuals choose schooling level to maximize lifetime utility
\end{itemize}
\end{frame}


% \begin{frame}\frametitle{}
% Now that we are modeling people's choices, we need to quantify preferences

% \begin{align}
% u_1\left(z,c,\eta_1\right) & = f\left(z,c,\eta_1\right) \nonumber \\
% u_2\left(w\left(s,x\right),k,\eta_2\right) & = g\left(w\left(s,x\right),k,\eta_2\right)
% \label{eq:utils}
% \end{align}

% \begin{itemize}
%     \item<2-> $z$ is family background
%     \item[]<2->
%     \item<2-> $c$ is schooling costs
%     \item[]<2->
%     \item<2-> $k$ is number of kids in adult household
%     \item[]<2->
%     \item<2-> $\eta_t$ are unobservable preferences [similar to $\varepsilon$ in equation \eqref{eq:basicmincer}]
% \end{itemize}

% \end{frame}

% \begin{frame}\frametitle{}
% And since we are working with a dynamic model, we need to write down the lifetime utility function---with discount factor $\delta \in [0,1)$:

% \begin{align}
% V & = u_1\left(z,c,\eta_1\right) + \delta u_2\left(w\left(s,x\right),k,\eta_2\right)
% \label{eq:PDV}
% \end{align}

% \begin{itemize}
%     \item<2-> Equations \eqref{eq:basicmincer}--\eqref{eq:PDV} define our model
%     \item[]<2->
%     \item<2-> This model is still \textbf{laughably unrealistic}, but at least we have something
%     \item[]<2->
%     \item<2-> A number of questions remain, but we'll ignore these for now
% \end{itemize}

% \end{frame}


% \begin{frame}\frametitle{}

% Once you write down the model, it's helpful to classify the different model objects:
% \bigskip{}


% \begin{columns}<2->
% \begin{column}{0.5\textwidth}
% \textbf{Exogenous variables}
% \begin{itemize}
%     \item family background $(z)$
%     \item schooling costs $(c)$
%     \item children in household $(k)$
% \end{itemize}

% \textbf{Endogenous variables}
% \begin{itemize}
%     \item schooling $(s)$
%     \item period-2 work decision
% \end{itemize}
% \end{column}
% \begin{column}{0.5\textwidth}
% \bigskip{}
% \bigskip{}

% \textbf{Outcome variable}
% \begin{itemize}
%     \item hourly wages $(w)$
% \end{itemize}

% \textbf{Unobservables}
% \begin{itemize}
%     \item log wages $(\varepsilon)$
%     \item preferences $(\eta_t)$
% \end{itemize}

% \textbf{Model parameters}
% \begin{itemize}
%     \item returns to human capital $(\beta)$
%     \item discount factor $(\delta)$
%     \item other parameters implied by $f(\cdot)$ and $g(\cdot)$
% \end{itemize}
% \end{column}
% \end{columns}

% \end{frame}


\end{document}