\documentclass[aspectratio=169]{beamer}

\usetheme{default}
\setbeamertemplate{navigation symbols}{}
\setbeamertemplate{enumerate item}{\color{navy}\arabic{enumi}.}
\setbeamertemplate{itemize item}{\color{black}\textbullet}
\setbeamertemplate{itemize subitem}{\color{black}\textbullet}
\usepackage{booktabs}
\usepackage{xcolor}
\usepackage{tikz}
\usetikzlibrary{shapes,arrows,positioning}
\definecolor{navy}{RGB}{0, 0, 128}
\definecolor{lightblue}{RGB}{230,240,250}
\definecolor{darkgreen}{RGB}{0,100,0}
\definecolor{lightgreen}{RGB}{230,250,230}
\newcommand{\highlight}[1]{\colorbox{lightblue}{$\displaystyle\textcolor{navy}{#1}$}}
\newcommand{\highlighttext}[1]{\colorbox{lightblue}{\textcolor{navy}{#1}}}
\newcommand{\highlightgreen}[1]{\colorbox{lightgreen}{$\displaystyle\textcolor{darkgreen}{#1}$}}

\usepackage{hyperref}
\hypersetup{
    colorlinks=true,
    linkcolor=navy,
    urlcolor=navy,
    citecolor=navy
}

\begin{document}

\begin{frame}

Factor Analysis comes in two forms: Exploratory (EFA) and Confirmatory (CFA)

\bigskip{}

\begin{itemize}
\itemsep1.5em
\item<2-> EFA: see what factors might be in the data

\item<3-> CFA: write down a model and use the data to test it
\end{itemize}

\bigskip{}

\onslide<4->{
In structural econometrics, we pretty much only do CFA
}

\bigskip{}

\end{frame}

\begin{frame}

FA is used extensively in psychometrics
\bigskip{}

\onslide<2->{
It is a natural tool for analyzing cognitive or behavioral tests
}

\bigskip{}

\begin{itemize}
\itemsep1.5em
\item<3-> Each test measures some set of skills, but does so noisily

\item<4-> Tests tend to measure the same set of skills, so they are correlated
\end{itemize}

\end{frame}

\begin{frame}

Suppose our $J$ columns of $M$ correspond to measurements (e.g. test scores)

\bigskip{}

\onslide<2->{
FA tries to find some underlying unobservables that commonly affect $M$
}

\bigskip{}

\onslide<3->{
We assume that we cannot observe $\boldsymbol{\theta}$
}

\bigskip{}

\onslide<4->{
If we assume that $M$ is standardized (mean-zero, unit-variance), then
\begin{align*}
M &= \underbrace{\boldsymbol{\theta}_k\Lambda_k + \boldsymbol{\varepsilon}}_{\boldsymbol{u}}
\end{align*}
}
\bigskip{}

\onslide<5->{
$\boldsymbol{u}$ is a composite error term (since both $\boldsymbol{\theta}$ and $\varepsilon$ are unobservable)
}
\bigskip{}

\onslide<6->{
In FA, we call $\boldsymbol{\theta}$ \textcolor{navy}{factors}, and we call $\Lambda$ \textcolor{navy}{factor loadings} and $\boldsymbol{\varepsilon}$ \textcolor{navy}{uniquenesses}
}

\end{frame}

\begin{frame}

Clearly, PCA and FA are related, but there are important differences

\bigskip{}

\begin{itemize}
\itemsep1.5em
\item<2-> When we drop components, both have error: $M = \boldsymbol{\theta}_k\Lambda_k + \boldsymbol{\varepsilon}$

\item<3-> But the $\boldsymbol{\theta}_k$, $\Lambda_k$, and $\boldsymbol{\varepsilon}$ are all different between PCA and FA

\item<4-> PCA error: approximation from dimensionality reduction ($\to 0$ as $K\to J$)

\item<5-> FA error: measurement error (persists even with all factors retained)
\end{itemize}
\bigskip{}

\onslide<6->{
FA separates common variance from unique variance; PCA explains total variance
}

\bigskip{}

\onslide<7->{
For many more excellent details, see \href{http://www.stat.cmu.edu/~cshalizi/uADA/12/lectures/ch19.pdf}{Shalizi (2019)}
}

\end{frame}


\begin{frame}

\onslide<1->{
In econometrics, we often want to allow covariates $X$ to affect our measurements
}

\bigskip{}

\onslide<2->{
We can extend FA to incorporate observable characteristics:
\begin{align*}
\only<2>{
M &= \highlight{X\boldsymbol{\beta}} + \underbrace{\boldsymbol{\theta}_k\Lambda_k + \boldsymbol{\varepsilon}}_{\boldsymbol{u}} \\
}
\onslide<3->{
M &= X\boldsymbol{\beta} + \underbrace{\boldsymbol{\theta}_k\Lambda_k + \boldsymbol{\varepsilon}}_{\boldsymbol{u}} \\
}
\onslide<3->{
  &= \underbrace{X}_{N\times L}\times\underbrace{\boldsymbol{\beta}}_{L\times J} + \underbrace{\boldsymbol{\theta}_k}_{N\times K} \times \underbrace{\Lambda_{k}}_{K\times J} + \underbrace{\boldsymbol{\varepsilon}}_{N\times J}}
\end{align*}
}

\begin{itemize}
\itemsep1.5em

\item<4-> The factors $\boldsymbol{\theta}_k$ now capture common variation \emph{after} conditioning on $X$

\item<5-> \textcolor{navy}{Key challenge:} We need additional assumptions for identification
\end{itemize}

\end{frame}


\end{document}