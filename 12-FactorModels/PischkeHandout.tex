\documentclass[12pt]{article}
\usepackage{amsmath}
\usepackage{amssymb}
\usepackage{setspace}
\usepackage[margin=1in]{geometry}
\usepackage{booktabs}
\usepackage{xcolor}
\usepackage{tikz}
\usepackage{graphicx}
\usetikzlibrary{shapes,arrows,positioning}
\definecolor{navy}{RGB}{0, 0, 128}
\definecolor{lightblue}{RGB}{230,240,250}
\definecolor{darkgreen}{RGB}{0,100,0}
\definecolor{lightgreen}{RGB}{230,250,230}
\newcommand{\highlight}[1]{\colorbox{lightblue}{$\displaystyle\textcolor{navy}{#1}$}}
\newcommand{\highlighttext}[1]{\colorbox{lightblue}{\textcolor{navy}{#1}}}
\newcommand{\highlightgreen}[1]{\colorbox{lightgreen}{$\displaystyle\textcolor{darkgreen}{#1}$}}

\usepackage{hyperref}
\hypersetup{
    colorlinks=true,
    linkcolor=navy,
    urlcolor=navy,
    citecolor=navy
}


\title{Lecture Notes on Measurement Error}
\author{Steve Pischke}
\date{Spring 2007}
\begin{document}
\maketitle
These notes summarize a variety of simple results on measurement error which I find useful. They also provide some references where more complete results and applications can be found.
\section*{Classical Measurement Error}
We will start with the simplest regression models with one independent variable. For expositional ease we also assume that both the dependent and the explanatory variable have mean zero. Suppose we wish to estimate the population relationship
\begin{equation}
\label{eq:1}
y = \beta x + \varepsilon
\end{equation}
Unfortunately, we only have data on
\begin{equation}
\label{eq:2}
\tilde{x} = x + u
\end{equation}
\begin{equation}
\label{eq:3}
\tilde{y} = y + v
\end{equation}
i.e. our observed variables are measured with an additive error. Let's make the following simplifying assumptions
\begin{equation}
\label{eq:4}
E(u) = 0
\end{equation}
\begin{equation}
\label{eq:5}
\text{plim } \frac{1}{n}(y'u) = 0
\end{equation}
\begin{equation}
\label{eq:6}
\text{plim } \frac{1}{n}(x'u) = 0
\end{equation}
\begin{equation}
\label{eq:7}
\text{plim } \frac{1}{n}(\varepsilon'u) = 0
\end{equation}
The measurement error in the explanatory variable has mean zero, is uncorrelated with the true dependent and independent variables and with the equation error. Also we will start by assuming $\sigma_v^2 = 0$, i.e. there is only measurement error in $x$. These assumptions define the classical errors-in-variables model.

Substitute \eqref{eq:2} into \eqref{eq:1}:
\begin{equation}
\label{eq:8}
y = \beta(\tilde{x} - u) + \varepsilon = y_i = \beta \tilde{x} + (\varepsilon - \beta u)
\end{equation}
The measurement error in $x$ becomes part of the error term in the regression equation thus creating an endogeneity bias. Since $\tilde{x}$ and $u$ are positively correlated (from \eqref{eq:2}) we can see that OLS estimation will lead to a negative bias in $\hat{\beta}$ if the true $\beta$ is positive and a positive bias if $\beta$ is negative.

To assess the \textbf{size of the bias} consider the OLS-estimator for $\beta$
\[
\hat{\beta} = \frac{\text{cov}(\tilde{x}, y)}{\text{var}(\tilde{x})} = \frac{\text{cov}(x + u, \beta x + \varepsilon)}{\text{var}(x + u)}
\]
and
\[
\text{plim } \hat{\beta} = \beta \frac{\sigma_x^2}{\sigma_x^2 + \sigma_u^2} = \beta \lambda
\]
where
\[
\lambda \equiv \frac{\sigma_x^2}{\sigma_x^2 + \sigma_u^2}
\]
The quantity $\lambda$ is referred to as reliability or signal-to-total variance ratio. Since $0 < \lambda < 1$ the coefficient $\hat{\beta}$ will be biased towards zero. This bias is therefore called attenuation bias and $\lambda$ is the attenuation factor in this case.


The bias is
\[
\text{plim } \hat{\beta} - \beta = \beta \lambda - \beta = -\beta(1 - \lambda) = -\beta \frac{\sigma_u^2}{\sigma_x^2 + \sigma_u^2}
\]
which again brings out the fact that the bias depends on the sign and size of $\beta$.

In order to figure out what happens to the estimated standard error first consider estimating the residual variance from the regression
\[
\hat{\varepsilon} = y - \hat{\beta}\tilde{x} = y - \hat{\beta}(x + u)
\]
Add and subtract the true error $\varepsilon = y - \beta x$ from this equation and collect terms.
\begin{align*}
\hat{\varepsilon} &= \varepsilon - (y - \beta x) + y - x\hat{\beta} - u\hat{\beta}\\
&= \varepsilon + (\beta - \hat{\beta})x - u\hat{\beta}
\end{align*}

You notice that the residual contains two additional sources of variation compared to the true error. The first is due to the fact that $\hat{\beta}$ is biased towards zero. Unlike in the absence of measurement error the term $\hat{\beta}$ does not vanish asymptotically. The second term is due to the additional variance introduced by the presence of measurement error in the regressor. Note that by assumption the three random variables $\varepsilon$, $x$, and $u$ in this equation are uncorrelated. We therefore obtain for the estimated variance of the equation error
\[
\text{plim } \hat{\sigma}_{\varepsilon}^2 = \sigma_{\varepsilon}^2 + (1 - \lambda)^2 \beta^2 \sigma_x^2 + \lambda^2 \beta^2 \sigma_u^2
\]

For the estimate of the variance of $\sqrt{n}(\hat{\beta} - \beta)$, call it $s_{\beta}$, we have
\begin{align*}
\text{plim } s_{\beta} &= \text{plim } \frac{\hat{\sigma}_{\varepsilon}^2}{\hat{\sigma}_{\tilde{x}}^2}\\
&= \frac{\sigma_{\varepsilon}^2 + (1 - \lambda)^2 \beta^2 \sigma_x^2 + \lambda^2 \beta^2 \sigma_u^2}{\sigma_x^2 + \sigma_u^2}\\
&= \frac{\sigma_x^2}{\sigma_x^2 + \sigma_u^2} \frac{\sigma_{\varepsilon}^2}{\sigma_x^2} + \frac{\sigma_x^2}{\sigma_x^2 + \sigma_u^2}(1 - \lambda)^2 \beta^2 + \frac{\sigma_u^2}{\sigma_x^2 + \sigma_u^2} \lambda^2 \beta^2\\
&= \lambda \frac{\sigma_{\varepsilon}^2}{\sigma_x^2} + \lambda(1 - \lambda)^2 \beta^2 + \lambda^2 (1 - \lambda) \beta^2\\
&= \lambda s + \lambda(1 - \lambda)^2 \beta^2
\end{align*}

The first term indicates that the true standard error is underestimated in proportion to $\lambda$. Since the second term is positive we cannot sign the overall bias in the estimated standard error.

However, the t-statistic will be biased downwards. The t-ratio converges to
\[
\text{plim } t\sqrt{n} = \frac{\text{plim } \hat{\beta}}{\text{plim } \sqrt{s_{\beta}}} = \frac{\beta \lambda}{\sqrt{\lambda s + \lambda(1 - \lambda)^2 \beta^2}} = \frac{\beta\sqrt{\lambda}}{\sqrt{s + (1 - \lambda)^2 \beta^2}}
\]
which is smaller than $\beta/\sqrt{s}$.

\section*{Simple Extensions}
Next, consider measurement error in the dependent variable $y$, i.e. let $\sigma_v^2 > 0$ while $\sigma_u^2 = 0$. Substitute \eqref{eq:3} into \eqref{eq:1}:
\[
\tilde{y} = \beta x + \varepsilon + v
\]


Since $v$ is uncorrelated with $x$ we can estimate $\beta$ consistently by OLS in this case. Of course, the estimates will be less precise than with perfect data.


Return to the case where there is measurement error only in $x$. The fact that measurement error in the dependent variable is more innocuous than measurement error in the independent variable might suggest that we run the reverse regression of $x$ on $y$ thus avoiding the bias from measurement error. Unfortunately, this does not solve the problem. Reverse \eqref{eq:8} to obtain
\[
\tilde{x} = \frac{1}{\beta}y - \frac{1}{\beta}\varepsilon + u
\]

$u$ and $y$ are uncorrelated by assumption but $y$ is correlated with the equation error $\varepsilon$ now. So we have cured the regression of errors-in-variables bias but created an endogeneity problem instead. Note, however, that this regression is still useful because $\varepsilon$ and $y$ are negatively correlated so that $\widehat{1/\beta}$ is biased downwards, implying an upward bias for $\hat{\beta}_r = \widehat{1/\beta}$. Thus the results from the standard regression and from the reverse regression will bracket the true coefficient, i.e. plim $\hat{\beta} < \beta < \text{plim } \hat{\beta}_r$. Implicitly, this bracketing result uses the fact that we know that $\sigma_{\varepsilon}^2$ and $\sigma_u^2$ have to be positive. The bounds of this interval are obtained whenever one of the two variances is zero. This implies that the interval tends to be large when these variances are large. In practice the bracketing result is therefore often not very informative. The bracketing result extends to multivariate regressions: in the case of two regressors you can run the original as well as two reverse regressions. The results will imply that the true $(\beta_1, \beta_2)$ lies inside the triangular area mapped out by these three regressions, and so forth for more regressors \cite{Klepper_Leamer_1984}.

Another useful fact to notice is that data transformations will typically magnify the measurement error problem. Assume you want to estimate the relationship
\[
y = \beta x + \gamma x^2 + \varepsilon
\]
Under normality the attenuation factor for $\hat{\gamma}$ will be the square of the attenuation factor for $\hat{\beta}$ \cite{Griliches_1986}.

So what can we do to get consistent estimates of $\beta$?

\begin{itemize}
\item If either $\sigma_x^2$, $\sigma_u^2$, or $\lambda$ is known we can make the appropriate adjustment for the bias in $\beta$. Either one of these is sufficient as we can estimate $\sigma_x^2 + \sigma_u^2$ (= plim var$(\tilde{x})$) consistently. Such information may come from validation studies of our data. In grouped data estimation, i.e. regression on cell means, the sampling error introduced by the fact that the means are calculated from a sample can be estimated \cite{Deaton_1985}. This only matters if cell sizes are small; grouped data estimation yields consistent estimates with cell sizes going to infinity (but not with the number of cells going to infinity at constant cell sizes).

\item Any instrument $z$ correlated with $x$ but uncorrelated with $u$ will identify the true coefficient since
\[
\hat{\beta}_{IV} = \frac{\text{cov}(y, z)}{\text{cov}(\tilde{x}, z)} = \frac{\text{cov}(\beta x + \varepsilon, z)}{\text{cov}(x + u, z)}
\]
\[
\text{plim } \hat{\beta}_{IV} = \beta \frac{\sigma_{xz}}{\sigma_{xz}} = \beta
\]

In this case it is also possible to get a consistent estimate of the population $R^2 = \beta^2 \sigma_x^2/\sigma_y^2$. The estimator
\[
\widetilde{R}^2 = \hat{\beta}_{IV} \frac{\text{cov}(y, \tilde{x})}{\text{var}(y)} = \hat{\beta}_{IV} \hat{\beta}_r
\]

which is the product of the IV coefficient and the OLS coefficient from the reverse regression, yields
\[
\text{plim } \widetilde{R}^2 = \beta^2 \frac{\sigma_x^2}{\sigma_y^2} = R^2
\]
\item Get better data.
\end{itemize}
\section*{Panel Data}
Often we are interested in using panel data to eliminate fixed effects. How does measurement error affect the fixed effects estimator? Extend the one variable model in \eqref{eq:1} to include a fixed effect:
\begin{equation}
\label{eq:9}
y_{it} = \beta x_{it} + \alpha_i + \varepsilon_{it}
\end{equation}
Difference this to eliminate the fixed effect $\alpha_i$.
\[
y_{it} - y_{it-1} = \beta(x_{it} - x_{it-1}) + \varepsilon_{it} - \varepsilon_{it-1}
\]

As before we only observe $x_{it}^{e} = x_{it} + u_{it}$. Using our results from above
\[
\text{plim } \hat{\beta} = \beta \frac{\sigma_{\Delta x}^2}{\sigma_{\Delta x}^2 + \sigma_{\Delta u}^2}
\]

So we have to figure out how the variance in the changes of $x$ relates to the variance in the levels.
\begin{align*}
\sigma_{\Delta x}^2 &= \text{var}(\Delta x_t) - 2\text{cov}(x_t, x_{t-1}) + \text{var}(x_{t-1})
\end{align*}

If the process for $x_t$ is stationary this simplifies to
\begin{align*}
\sigma_{\Delta x}^2 &= 2\sigma_x^2 - 2\text{cov}(x_t, x_{t-1})\\
&= 2\sigma_x^2(1 - \rho)
\end{align*}

where $\rho$ is the first order autocorrelation coefficient in $x_t$. Similarly, define $r$ to be the autocorrelation coefficient in $u_t$ so we can write
\[
\text{plim } \hat{\beta} = \beta \frac{\sigma_x^2(1 - \rho)}{\sigma_x^2(1 - \rho) + \sigma_u^2(1 - r)} = \beta \frac{1}{1 + \frac{\sigma_u^2(1-r)}{\sigma_x^2(1-\rho)}}
\]

In the special case where both $x_t$ and $u_t$ are uncorrelated over time the attenuation bias for the fixed effects estimator simplifies to the original $\lambda$. Fixed effects estimation is particularly worrisome when $r = 0$, i.e. the measurement error is just serially uncorrelated noise, while the signal is highly correlated over time. In this case, differencing doubles the variance of the measurement error while it might reduce the variance of the signal. In the effort to eliminate the bias arising from the fixed effect we have introduced additional bias due to measurement error. Of course, differencing is highly desirable if the measurement error $u_{it} = u_i$ is a fixed effect itself. In this case differencing eliminates the measurement error completely. In general, differencing is desirable when $r > \rho$. For panel earnings data $\rho \approx 2r$ \cite{Bound_etal_1994, Bound_Krueger_1991}.

Sometimes it is reasonable to make specific assumptions about the behavior of the measurement error over time. For example, if we are willing to assume that $u_{it}$ is i.i.d. while the $x$'s are correlated then it is possible to identify the true $\beta$ even in relatively short panels. The simplest way to think about this is in a four period panel. Form differences between the third and second period and instrument these with differences between the fourth and the first period. Obviously
\[
\text{plim } \frac{1}{n}(u_4 - u_1)'(u_3 - u_2) = 0
\]

by the i.i.d. assumption for $u_{it}$. The long and short differences for $x_{it}$ will be correlated, on the other hand, since the $x$'s are correlated over time. We have constructed a valid instrument. This example makes much stronger assumptions than are necessary. Alternatively, with four periods and the i.i.d. assumption for $u_{it}$ we can come up with much more efficient estimators since other valid instruments can be constructed \cite{Griliches_Hausman_1986}. They also point out that comparing the results from first difference estimates, long difference estimates, and deviations from means estimates provides a useful test for measurement error if $\rho \neq r$ since the attenuation bias varies depending on the specific estimator chosen. But be aware that the same happens if your model is misspecified in some other way, for example there are neglected true dynamics in your $x$'s, so your test only indicates ``some misspecification.''

\section*{Multivariate Models}
Return to OLS estimation in a simple cross-section and consider what happens to the bias as we add more variables to the model. Consider the equation
\begin{equation}
\label{eq:10}
y = \beta x + \gamma w + \varepsilon
\end{equation}
Even if only $\tilde{x}$ is subject to measurement error while $w$ is measured correctly both parameters will in general be biased now. $\hat{\gamma}$ is unbiased when the two regressors are uncorrelated. $\hat{\beta}$ is still biased towards zero. We can also determine how the bias in $\hat{\beta}$ in the multivariate regression is related to the attenuation bias in the bivariate regression (which may also suffer from omitted variable bias now). To figure this out, consider the formula for $\hat{\beta}$ in the two variable case
\[
\hat{\beta} = \frac{\text{var}(w)\text{cov}(y, \tilde{x}) - \text{cov}(w, \tilde{x})\text{cov}(y, w)}{\text{var}(\tilde{x})\text{var}(w) - \text{cov}(w, \tilde{x})^2}
\]

Thus we obtain
\begin{align*}
\text{plim } \hat{\beta} &= \frac{\sigma_w^2(\beta\sigma_x^2 + \gamma\sigma_{xw}) - \sigma_{xw}^{e}(\beta\sigma_w^2 + \gamma\sigma_{xw})}{\sigma_w^2(\sigma_x^2 + \sigma_u^2) - (\sigma_{xw}^{e})^2}\
&= \beta \frac{\sigma_w^2\sigma_x^2 - \sigma_{xw}^{e}\sigma_{xw} + \sigma_w^2(\sigma_{xw} - \sigma_{xw}^{e})}{\sigma_w^2(\sigma_x^2 + \sigma_u^2) - (\sigma_{xw}^{e})^2}
\end{align*}
This does not get us much further. However, in the special case where $w$ is only correlated with $x$ but not with $u$, this can be simplified because now $\sigma_{xw} = \sigma_{xw}^{e}$ so that
\begin{equation}
\label{eq:11}
\text{plim } \hat{\beta} = \beta \frac{\sigma_w^2\sigma_x^2 - (\sigma_{xw})^2}{\sigma_w^2(\sigma_x^2 + \sigma_u^2) - (\sigma_{xw})^2} = \beta\lambda'
\end{equation}

Notice that $\sigma_w^2\sigma_x^2 > (\sigma_{xw})^2$ which proves that $\hat{\beta}$ is biased towards zero.

There are various ways to rewrite \eqref{eq:11}. I find it instructive to look at the representation of the attenuation factor $\lambda'$ in terms of the reliability ratio $\lambda$ and the $R^2$ of a regression of $\tilde{x}$ on $w$. Since this is a one variable regression the population $R^2$ is just the square of the correlation coefficient of the variables
\[
R_{\tilde{x}w}^2 = \frac{(\sigma_{xw})^2}{\sigma_w^2(\sigma_x^2 + \sigma_u^2)}
\]

Dividing numerator and denominator in \eqref{eq:11} by $(\sigma_x^2 + \sigma_u^2)$ yields the following expression for the attenuation factor
\[
\lambda' = \frac{\frac{\sigma_w^2}{\sigma_x^2 + \sigma_u^2} - \frac{\sigma_w^2}{\sigma_x^2 + \sigma_u^2}R_{\tilde{x}w}^2}{\frac{\sigma_w^2}{\sigma_x^2 + \sigma_u^2} - \frac{\sigma_w^2}{\sigma_x^2 + \sigma_u^2}R_{\tilde{x}w}^2} = \frac{\lambda - R_{\tilde{x}w}^2}{1 - R_{\tilde{x}w}^2}
\]

This formula is quite intuitive. It says the following: if there is no omitted variable bias from estimating \eqref{eq:1} instead of \eqref{eq:10} because the true $\gamma = 0$, then the attenuation bias will increase as additional regressors (correlated with $x$) are added since the expression above is decreasing in $R_{\tilde{x}w}^2$. What is going on is that the additional regressor $w$ will now serve as a proxy for part of the signal in $x$. Therefore, the partial correlation between $y$ and $\tilde{x}$ will be attenuated more, since some of the signal has been taken care of by the $w$ already. Notice that $R_{\tilde{x}w}^2 < \lambda$ because $w$ is only correlated with $x$ but not with $u$. Hence $0 < \lambda' < \lambda < 1$.

In the special case just discussed, and if $x$ and $w$ are positively correlated, the bias in $\hat{\gamma}$ will have the opposite sign of the bias in $\hat{\beta}$. In fact, with the additional assumption that $\sigma_x^2 = \sigma_w^2$ we have
\[
\text{plim } \hat{\gamma} - \gamma = \rho_{xw}\beta(1 - \lambda') = \rho_{xw}\beta(\text{plim } \hat{\beta} - \beta)
\]

where $\rho_{xw}$ is the correlation coefficient between $x$ and $w$.

When $\gamma \neq 0$, comparisons between the bivariate regression of $y$ on $\tilde{x}$ and the multivariate model including $w$ are harder to interpret because we have to keep in mind that the bivariate regression is now also subject to omitted variable bias. Some results are available for special cases. If $\beta > 0$, $\gamma > 0$ and $x$ and $w$ are positively correlated (but $w$ is still uncorrelated with $u$) then the probability limit of the estimated $\hat{\beta}$ in the multivariate regression will be lower than in the bivariate regression \cite{Maddala_1977}. This follows because adding $w$ to the regression purges it of the (positive) omitted variable bias while introducing additional (negative) attenuation bias. This example also makes it clear that no such statements will be possible if the omitted variable bias is negative.

\section*{Non-classical Measurement Error}
We will now start relaxing the classical assumptions. Return to the model \eqref{eq:1} and \eqref{eq:2} but drop assumption \eqref{eq:6} that $x$ and $u$ are uncorrelated. Recall that
\[
\hat{\beta} = \frac{\text{cov}(x + u, \beta x + \varepsilon)}{\text{var}(x + u)}
\]

so that we have in this case
\begin{equation}
\label{eq:12}
\text{plim } \hat{\beta} = \beta \frac{(\sigma_x^2 + \sigma_{xu})}{\sigma_x^2 + \sigma_u^2 + 2\sigma_{xu}} = \beta\left[1 - \frac{(\sigma_u^2 + \sigma_{xu})}{\sigma_x^2 + \sigma_u^2 + 2\sigma_{xu}}\right] = \beta(1 - b_{u\tilde{x}})
\end{equation}
Notice that the numerator in $b_{u\tilde{x}}$ is the covariance between $\tilde{x}$ and $u$. Thus, $b_{u\tilde{x}}$ is the regression coefficient of a regression of $u$ on $\tilde{x}$. The classical case is a special case of this where this regression coefficient $b_{u\tilde{x}} = 1$. The derivative of $1 - b_{u\tilde{x}}$ with respect to $\sigma_{xu}$ has the sign of $\sigma_u^2 - \sigma_x^2$: Starting from a situation where $\sigma_{xu} = 0$ (classical measurement error) increasing this covariance increases the attenuation factor (decreases the bias) if more than half of the variance in $\tilde{x}$ is measurement error and decreases it otherwise. In earnings data this covariance tends to be negative \cite{Bound_Krueger_1991}. If $\tilde{x}$ consisted mostly of measurement error then a more negative $\sigma_{xu}$ implies a lower attenuation factor and may even reverse the sign of the estimated $\beta$.

Measurement error in the dependent variable that is correlated with the true $y$ or with the $x$'s can be analyzed along similar lines. A general framework for this is provided by \cite{Bound_etal_1994}. Make $X$ an $n \times k$ matrix of covariates, $\beta$ a $k$ vector of coefficients, etc. so that \eqref{eq:1} becomes
\[
y = X\beta + \varepsilon
\]

Then
\begin{align*}
\hat{\beta} &= (\tilde{X}'\tilde{X})^{-1}\tilde{X}'\tilde{y} = (\tilde{X}'\tilde{X})^{-1}\tilde{X}'(\tilde{X}\beta - U\beta + v + \varepsilon)\
&= \beta + (\tilde{X}'\tilde{X})^{-1}\tilde{X}'(-U\beta + v + \varepsilon)
\end{align*}
and
\[
\text{plim } \hat{\beta} = \beta + \text{plim}(\tilde{X}'\tilde{X})^{-1}\tilde{X}'(-U\beta + v)
\]
Collecting the measurement errors in a matrix
\[
W = [U \mid v]
\]
yields
\begin{equation}
\label{eq:13}
\text{plim } \hat{\beta} = \beta + \text{plim}(\tilde{X}'\tilde{x})^{-1}\tilde{X}'W\begin{bmatrix} -\beta \ 1 \end{bmatrix}
\end{equation}
so that the biases in more general cases can always be thought of in terms of regression coefficients from regressing the measurement errors on the mismeasured covariates. Special cases like \eqref{eq:12} are easily obtained from \eqref{eq:13}. These regression coefficients of the measurement errors on the mismeasured covariates are therefore what validation studies ought to focus on.
What happens when we do instrumental variables in this case? For simplicity, focus on the one regressor case.
\[
\hat{\beta}_{IV} = \frac{\text{cov}(y, z)}{\text{cov}(\tilde{x}, z)} = \frac{\text{cov}(\beta x + \varepsilon, z)}{\text{cov}(x + u, z)}
\]
\[
\text{plim } \hat{\beta}_{IV} = \beta \frac{\sigma_{xz}}{\sigma_{xz} + \sigma_{zu}}
\]
This demonstrates that we can still get consistent estimates by using instrumental variables as long as the instruments are only correlated with true $X$'s but not with any of the measurement errors, i.e. the term $\sigma_{zu} = 0$ above. On the other hand, this condition is much more challenging in this case, since we have $\sigma_{xu} \neq 0$ and we need $\sigma_{zu} = 0$ and $\sigma_{zx} \neq 0$. Think, for example, about the case where $z = x + \eta$ is a second independent report of the same underlying $x$. In this case, $\sigma_{zu} = \sigma_{xu} + \sigma_{u\eta}$. Hence, even if the errors were uncorrelated, i.e. $\sigma_{u\eta} = 0$, we still have $\sigma_{zu} = \sigma_{xu} \neq 0$ \cite{Black_etal_1998}. The upshot from this is that the instruments most likely to be helpful are the types of instruments we would be using anyway for other reasons (say to cure selection bias). For example, quarter of birth in \cite{Angrist_Krueger_1991} is much less likely to be correlated with the measurement error in schooling than is a sibling's report of ones schooling \cite{Ashenfelter_Krueger_1994}.

\section*{Special Cases of Non-classical Error: Group Aggregates and Optimal Predictors}
Occasionally, we run into the following problem. We wish to estimate the standard regression model
\[
y_{it} = \beta x_{it} + \varepsilon_{it}
\]
but instead of $x_{it}$ we only observe the group or time average $\bar{x}_t$. For example, we may wish to estimate a wage curve, where $y_{it}$ are individual level wages over time and $\bar{x}_t$ is the aggregate unemployment rate, or $x_{it}$ might be class size in school, but you only know class size at the school level and not at the individual level. Obviously, $\bar{x}_t$ is an error ridden version of the true regressor $x_{it}$. Typically, $\bar{x}_t$ will be the mean of $x_{it}$, often from a larger sample or in the population so that $x_{it} = \bar{x}_t + u_{it}$, and $u_{it}$ will be uncorrelated with $\bar{x}_t$. If this is the case, the OLS estimator of $\beta$ is consistent. It is easy to see that this is true:
\begin{align*}
\hat{\beta} &= \frac{\text{cov}(y_{it}, \bar{x}_t)}{\text{var}(\bar{x}_t)} = \frac{\text{cov}(\beta x_{it} + \varepsilon_{it}, \bar{x}_t)}{\text{var}(\bar{x}_t)}\\
&= \frac{\text{cov}\left(\beta(\bar{x}_t + u_{it}) + \varepsilon_{it}, \bar{x}_t\right)}{\text{var}(\bar{x}_t)}
\end{align*}

so that
\[
\text{plim } \hat{\beta} = \beta \frac{\sigma_{\bar{x}t}^2}{\sigma{\bar{x}_t}^2} = \beta
\]
While this looks similar to a classical measurement error problem, it is not. In the classical case the observed regressor $\bar{x}_t$ equals the true regressor plus noise that is uncorrelated with the truth. Here, the true regressor $x_{it}$ equals the observed regressor plus noise that is uncorrelated with the observed regressor. In terms of the notation we developed above, the covariance between the true $x$ and the measurement error $\sigma_{xu} = -\sigma_u^2$. The negative covariance of the measurement error with the true regressor just cancels the effect of the measurement error, or $b_{u\tilde{x}} = 0$ in \eqref{eq:12}. Therefore, our estimates are consistent. Moreover, OLS using the group average will yield correct standard errors. These will be larger, of course, than in the case where the micro level regressor $x_{it}$ is available.

The point here is of course not limited to group aggregates. Whenever the measurement error $u$ is uncorrelated with the mismeasured regressor $\tilde{x}$, the coefficient $\beta$ can be estimated consistently by OLS. This can easily be seen from inspection of \eqref{eq:8}: $u$ is part of the error term but it is uncorrelated with the regressor $\tilde{x}$, hence there is no bias. This type of measurement error will arise when the reported variable $\tilde{x}$ is an optimal predictor of the true $x$ as demonstrated by \cite{Hyslop_Imbens_2001}. An optimal predictor has the property $\tilde{x} = E(x \mid \text{information set})$: From this it follows that $x = \tilde{x} + u$ with $u \perp \tilde{x}$, i.e. the prediction error is orthogonal to the predictor. The optimal predictor case would arise if respondents in surveys realize that they may have imperfect information about the variable they are asked to report. Because of this they downweight the error ridden information they have by the appropriate amount.

It turns out that well intentioned IV estimation might lead to biases in this case. Suppose you have two independent measures of $x$:
\begin{align*}
\tilde{x}_1 &= x + u_1\\
\tilde{x}_2 &= x + u_2
\end{align*}

where the $u_j$'s are classical measurement errors. The optimal predictors of $x$ are
\begin{equation}
\label{eq:14}
\begin{aligned}
x_1 &= E(x \mid \tilde{x}_1) = (1 - \lambda_1)x + \lambda_1 \tilde{x}_1\\
x_2 &= E(x \mid \tilde{x}_2) = (1 - \lambda_2)x + \lambda_2 \tilde{x}_2
\end{aligned}
\end{equation}

where
\[
\lambda_j \equiv \frac{\sigma_x^2}{\sigma_x^2 + \sigma_{u_j}^2}
\]
are the corresponding reliabilities. Then the IV estimator with $x_1^{*}$ as the endogenous regressor and $x_2^{*}$ as the instrument is
\begin{align*}
\hat{\beta}_{IV} &= \frac{\text{cov}(y, x_2^{*})}{\text{cov}(x_1^{*}, x_2^{*})} = \frac{\text{cov}(\beta x + \varepsilon, (1 - \lambda_2)x + \lambda_2 \tilde{x}_1)}{\text{cov}((1 - \lambda_1)x + \lambda_1 \tilde{x}_1, (1 - \lambda_2)x + \lambda_2 \tilde{x}_2)}\\
\text{plim}\hat{\beta}_{IV} &= \beta \frac{\lambda_2\sigma_x^2}{\lambda_1\lambda_2\sigma_x^2} = \frac{\beta}{\lambda_1}
\end{align*}

This demonstrates that the IV estimator will be biased up. There will be no way of telling whether measurement error is of the classical or of the optimal predictor type from the comparison of the OLS and IV estimates. The OLS estimate will always be attenuated relative to the IV estimate. But in the classical case the IV estimate is the consistent one, and in the optimal predictor case it is the OLS one.

Finally note that measurement error of this type will also lead to bias when $x$ is measured correctly but $y$ is measured with error, and the error is of the optimal predictor type. As in \eqref{eq:14}, the optimal predictor of $y$ will downweight the $\tilde{y}$ by an attenuation factor (although a more complicated one in this case, see \cite{Hyslop_Imbens_2001} for details). But we know that the regression of $\tilde{y}$ on $x$ will give unbiased estimates. So downweighting $\tilde{y}$ will lead to an estimate of $\beta$ which is attenuated.

For the case of earnings we know from \cite{Bound_etal_1994} that measurement error in earnings seems to be neither of the classical nor of the optimal predictor type because the error is correlated with both the reported and the true amounts.

\section*{Measurement Error in Dummy Variables}
There is an interesting special case of non-classical measurement error: that of a binary regressor. Obviously, misclassification of a dummy variable cannot lead to classical measurement error. If the dummy is one, measurement error can only be negative; if the dummy is zero, it can only be positive. So the measurement error is negatively correlated with the true variable. This problem has enough structure that it is worthwhile looking at it separately. Consider the regression
\begin{equation}
\label{eq:15}
y_i = \alpha + \beta d_i + \varepsilon_i
\end{equation}
where $d_i \in \{0, 1\}$. For concreteness, think of $y_i$ as wages, $d_i = 1$ as union members and $d_i = 0$ as nonmembers so that $\beta$ is the union wage differential. It is useful to note that the OLS estimate of $\beta$ is the difference between the mean of $y_i$ as $d_i = 1$ and the mean as $d_i = 0$. Instead of $d$ we observe a variable $d^{e}$ that misclassifies some observations. Take expectations of \eqref{eq:15} conditional on the observed value of $d_i$:
\begin{align*}
E(y_i \mid \tilde{d}_i = 1) &= \alpha + \beta P(d_i = 1 \mid \tilde{d}_i = 1)\\
E(y_i \mid \tilde{d}_i = 0) &= \alpha + \beta P(d_i = 1 \mid \tilde{d}_i = 0)
\end{align*}

The regression coefficient for the union wage differential is the sample analogue of the difference between these two, so it satisfies
\[
\text{plim } \hat{\beta} = \beta\left[P(d_i = 1 \mid \tilde{d}_i = 1) - P(d_i = 1 \mid \tilde{d}_i = 0)\right]
\]
Equation \eqref{eq:15} says that $\beta$ will be attenuated because some (high wage) union members are classified as nonmembers while some (low wage) nonmembers are classified as members.

We need some further notation. Let $q_1$ be the probability that we observe somebody to be a union member when he truly is, i.e. $q_1 \equiv P(\tilde{d}_i = 1 \mid d_i = 1)$, and similarly $q_0 \equiv P(\tilde{d}_i = 1 \mid d_i = 0)$. Thus $1 - q_1$ is the probability that a member is misclassified and $q_0$ is the probability that a nonmember is misclassified. Furthermore, let $\pi \equiv P(d_i = 1)$ be the true membership rate. Notice that the estimate of $\pi$ given by $\hat{\pi} = N^{-1}\sum \tilde{d}_i$ satisfies
\[
\text{plim } \hat{\pi} = q_1\pi + (1 - \pi)q_0
\]

Return to equation \eqref{eq:15}. By Bayes' Rule we can write the terms that appear in this equation as
\[
P(d_i = 1 \mid \tilde{d}_i = 1) = \frac{P(\tilde{d}_i = 1 \mid d_i = 1) \cdot P(d_i = 1)}{P(\tilde{d}_i = 1)} = \frac{q_1\pi}{q_1\pi + (1 - \pi)q_0}
\]
and
\[
P(d_i = 1 \mid \tilde{d}_i = 0) = \frac{(1 - q_1)\pi}{(1 - q_1)\pi + (1 - \pi)(1 - q_0)}
\]
and substituting back into \eqref{eq:15} yields
\begin{align}
\label{eq:16}
\text{plim } \hat{\beta} &= \beta\left[\frac{q_1\pi}{q_1\pi + (1 - \pi)q_0} - \frac{(1 - q_1)\pi}{(1 - q_1)\pi + (1 - \pi)(1 - q_0)}\right]\nonumber\\
&= \beta\left[\frac{q_1\pi}{\hat{\pi}} - \frac{(1 - q_1)\pi}{1 - \hat{\pi}}\right]\nonumber\\
&= \beta \frac{(1 - \hat{\pi})q_1\pi - \hat{\pi}(1 - q_1)\pi}{\hat{\pi}(1 - \hat{\pi})}\\
&= \beta \frac{\pi[(1 - \hat{\pi})q_1 - \hat{\pi}(1 - q_1)]}{\hat{\pi}(1 - \hat{\pi})}\nonumber\\
&= \beta \frac{\pi(q_1 - \hat{\pi})}{\hat{\pi}(1 - \hat{\pi})}\nonumber
\end{align}
Absent knowledge about $q_1$ and $q_0$ we cannot identify the true $\beta$ and $\pi$ from our data, i.e. from the estimates $\hat{\beta}$ and $\hat{\pi}$. In a multivariate regression, no simple formula like \eqref{eq:16} is available, although $\beta$ and $\pi$ can still be identified if $q_1$ and $q_0$ are known \cite{Aigner_1973}.

If we have panel data available and we are willing to impose the restriction that $q_1$ and $q_0$ do not change over time all coefficients will be identified. In fact, even with just a two period panel there is already one overidentifying restriction. To see this, notice that there are now not two states for union status but four possible transitions (continuous union members, continuous nonmembers, union entrants and union leavers). The key is that there have to be some switchers in the data. Then we can observe separate changes in $y$ over time for each of the four transition groups. Furthermore, we observe three independent transition probabilities. This makes a total of seven moments calculated from the data. From these we have to identify $\beta$, $q_1$, $q_0$, and the three true transition probabilities, i.e. only six parameters. The algebra is much messier \cite{Card_1996}. See \cite{Krueger_Summers_1988} for results on measurement error in multinomial variables (e.g. industry classifications).

\subsection*{Instrumental Variables Estimation of the Dummy Variable Model}
Suppose we have another binary variable $z_i$ available, which has the same properties as the mismeasured dummy variable $\tilde{d}_i$. Can we use $z_i$ as an instrument in the estimation of \eqref{eq:15}? Instrumental variables estimation will not yield a consistent estimate of $\beta$ in this case. The reason for this is simple. Recall that the measurement error can only be either $-1$ or $0$ (when $d_i = 1$), or $1$ or $0$ (when $d_i = 0$). This means that the measurement errors in two mismeasured variables will be positively correlated.

In order to study this case, define $h_1 \equiv P(z_i = 1 \mid d_i = 1)$ and $h_0 \equiv P(z_i = 1 \mid d_i = 0)$. The IV estimator in this case is simply the Wald estimator so that
\begin{equation}
\label{eq:17}
\text{plim } \hat{\beta}_{IV} = \frac{E(y_i \mid z_i = 1) - E(y_i \mid z_i = 0)}{E(\tilde{d}_i \mid z_i = 1) - E(\tilde{d}_i \mid z_i = 0)}.
\end{equation}

The numerator has the same form as \eqref{eq:15} with $z_i$ replacing $\tilde{d}_i$. The terms in the denominator can also easily be derived:
\begin{align*}
E(\tilde{d}_i \mid z_i = 1) &= P(\tilde{d}_i = 1 \mid z_i = 1)\\
&= \frac{P(\tilde{d}_i = 1, z_i = 1)}{P(z_i = 1)}\\
&= \frac{P(\tilde{d}_i = 1, z_i = 1 \mid d_i = 1)P(d_i = 1) + P(\tilde{d}_i = 1, z_i = 1 \mid d_i = 0)P(d_i = 0)}{P(z_i = 1 \mid d_i = 1)P(d_i = 1) + P(z_i = 1 \mid d_i = 0)P(d_i = 0)}\\
&= \frac{q_1h_1\pi + q_0h_0(1 - \pi)}{h_1\pi + h_0(1 - \pi)}
\end{align*}
and similarly for $E(\tilde{d}_i \mid z_i = 0)$. Substituting everything into \eqref{eq:17} yields
\[
\text{plim } \hat{\beta}_{IV} = \frac{\beta\left[\frac{h_1\pi}{h_1\pi + h_0(1-\pi)} - \frac{(1-h_1)\pi}{(1-h_1)\pi + (1-h_0)(1-\pi)}\right]}{\frac{q_1h_1\pi + q_0h_0(1-\pi)}{h_1\pi + h_0(1-\pi)} - \frac{q_1(1-h_1)\pi + q_0(1-h_0)(1-\pi)}{(1-h_1)\pi + (1-h_0)(1-\pi)}}.
\]

With some elementary algebra this simplifies to
\[
\text{plim } \hat{\beta}_{IV} = \beta \frac{\pi}{q_1 - q_0}.
\]
The IV estimate of $\beta$ is biased by a factor $1/(q_1 - q_0)$). This has some interesting features. The bias only depends on the misclassification rates in the variable $\tilde{d}_i$ which is being used as the endogenous regressor. This is because more misclassification in the instrument will lead to a smaller first stage coefficient. Since generally $1 > q_1 > q_0 > 0$, IV will be biased upwards. Hence, OLS and IV estimation could be used to bound the true coefficient.

However, the true coefficient is actually identified from the data, using an idea analogous to the panel data case above \cite{Kane_etal_1999}. There are seven sample moments which can be computed from the data. There are four cells defined by the cross-tabulation of $\tilde{d}_i$ and $z_i$. The mean of $y_i$ can be computed for each of these cells. In addition, we have three independent sampling fractions for the cross-tabulation. This makes a total of seven empirical moments. From these moments we have to identify $\alpha$, $\beta$, $\pi$, $q_0$, $q_1$, $h_0$, and $h_1$, i.e. seven parameters. These parameters are indeed just identified and can be estimated by method of moments.

\bibliographystyle{plainnat}
\begin{thebibliography}{99}
\bibitem[Aigner(1973)]{Aigner_1973}
Aigner, Dennis J. (1973). ``Regression with a Binary Independent Variable Subject to Errors of Observation.'' \textit{Journal of Econometrics} 1\eqref{eq:1}: 49--60.
\bibitem[Angrist and Krueger(1991)]{Angrist_Krueger_1991}
Angrist, Joshua D., and Alan B. Krueger (1991). ``Does Compulsory School Attendance Affect Schooling and Earnings?'' \textit{Quarterly Journal of Economics} 106\eqref{eq:4}: 979--1014.
\bibitem[Ashenfelter and Krueger(1994)]{Ashenfelter_Krueger_1994}
Ashenfelter, Orley, and Alan Krueger (1994). ``Estimates of the Economic Return to Schooling from a New Sample of Twins.'' \textit{American Economic Review} 84\eqref{eq:5}: 1157--1173.
\bibitem[Black, Berger, and Scott(1998)]{Black_etal_1998}
Black, Dan A., Mark C. Berger, and Frank A. Scott (1998). ``Bounding Parameter Estimates with Nonclassical Measurement Error.'' \textit{Journal of the American Statistical Association} 95(451): 739--748.
\bibitem[Bound et al.(1994)]{Bound_etal_1994}
Bound, John, Charles Brown, Greg J. Duncan, and Willard L. Rodgers (1994). ``Evidence on the Validity of Cross-Sectional and Longitudinal Labor Market Data.'' \textit{Journal of Labor Economics} 12\eqref{eq:3}: 345--368.
\bibitem[Bound and Krueger(1991)]{Bound_Krueger_1991}
Bound, John, and Alan B. Krueger (1991). ``The Extent of Measurement Error in Longitudinal Earnings Data: Do Two Wrongs Make a Right?'' \textit{Journal of Labor Economics} 9\eqref{eq:1}: 1--24.
\bibitem[Card(1996)]{Card_1996}
Card, David (1996). ``The Effect of Unions on the Structure of Wages: A Longitudinal Analysis.'' \textit{Econometrica} 64\eqref{eq:4}: 957--979.
\bibitem[Deaton(1985)]{Deaton_1985}
Deaton, Angus (1985). ``Panel Data from Time Series of Cross-Sections.'' \textit{Journal of Econometrics} 30(1--2): 109--126.
\bibitem[Griliches(1986)]{Griliches_1986}
Griliches, Zvi (1986). ``Data Problems in Econometrics.'' In Zvi Griliches and Michael D. Intriligator, eds., \textit{Handbook of Econometrics}, vol. 3. Amsterdam: North Holland, 1465--1514.
\bibitem[Griliches and Hausman(1986)]{Griliches_Hausman_1986}
Griliches, Zvi, and Jerry A. Hausman (1986). ``Errors in Variables in Panel Data.'' \textit{Journal of Econometrics} 31\eqref{eq:1}: 93--118.
\bibitem[Hyslop and Imbens(2001)]{Hyslop_Imbens_2001}
Hyslop, Dean R., and Guido W. Imbens (2001). ``Bias from Classical and Other Forms of Measurement Error.'' \textit{Journal of Business and Economic Statistics} 19\eqref{eq:4}: 475--481.
\bibitem[Kane, Rouse, and Staiger(1999)]{Kane_etal_1999}
Kane, Thomas J., Cecilia E. Rouse, and Douglas Staiger (1999). ``Estimating Returns to Schooling When Schooling Is Misreported.'' NBER Working Paper No. 7235.
\bibitem[Klepper and Leamer(1984)]{Klepper_Leamer_1984}
Klepper, Steven, and Edward E. Leamer (1984). ``Consistent Sets of Estimates for Regressions with Errors in All Variables.'' \textit{Econometrica} 52\eqref{eq:1}: 163--183.
\bibitem[Krueger and Summers(1988)]{Krueger_Summers_1988}
Krueger, Alan B., and Lawrence H. Summers (1988). ``Efficiency Wages and the Inter-Industry Wage Structure.'' \textit{Econometrica} 56\eqref{eq:2}: 259--293.
\bibitem[Maddala(1977)]{Maddala_1977}
Maddala, G. S. (1977). \textit{Econometrics}. New York: McGraw-Hill.
\end{thebibliography}
\end{document}