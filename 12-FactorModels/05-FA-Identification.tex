\documentclass[aspectratio=169]{beamer}

\usetheme{default}
\setbeamertemplate{navigation symbols}{}
\setbeamertemplate{enumerate item}{\color{navy}\arabic{enumi}.}
\setbeamertemplate{itemize item}{\color{black}\textbullet}
\setbeamertemplate{itemize subitem}{\color{black}\textbullet}
\usepackage{booktabs}
\usepackage{xcolor}
\usepackage{tikz}
\usetikzlibrary{shapes,arrows,positioning}
\definecolor{navy}{RGB}{0, 0, 128}
\definecolor{lightblue}{RGB}{230,240,250}
\definecolor{darkgreen}{RGB}{0,100,0}
\definecolor{lightgreen}{RGB}{230,250,230}
\newcommand{\highlight}[1]{\colorbox{lightblue}{$\displaystyle\textcolor{navy}{#1}$}}
\newcommand{\highlighttext}[1]{\colorbox{lightblue}{\textcolor{navy}{#1}}}
\newcommand{\highlightgreen}[1]{\colorbox{lightgreen}{$\displaystyle\textcolor{darkgreen}{#1}$}}

\usepackage{hyperref}
\hypersetup{
    colorlinks=true,
    linkcolor=navy,
    urlcolor=navy,
    citecolor=navy
}

\begin{document}

\begin{frame}

Recall our factor model with observable characteristics:
\begin{align*}
M &= X\boldsymbol{\beta} + \underbrace{\boldsymbol{\theta}\Lambda + \boldsymbol{\varepsilon}}_{\boldsymbol{u}} \\
\end{align*}
\only<2>{
We need to make the following assumptions:
\begin{align*}
\mathbb{E}\left(\boldsymbol{\varepsilon}\right) &= \mathbf{0}_{J\times 1}\\
\mathbb{V}\left(\boldsymbol{\varepsilon}\right) &\equiv \mathbb{E}\left(\boldsymbol{\varepsilon}'\boldsymbol{\varepsilon}\right)=\Omega_{J\times J}\\
\Omega_{[j,j]} &= \sigma^2_j\\
\Omega_{[j,k]} &= 0\\
\mathbb{E}\left(\boldsymbol{\theta}\right) &= \mathbf{0}_{2\times 1}\\
\mathbb{V}\left(\boldsymbol{\theta}\right) &= \Sigma_{\boldsymbol{\theta}}
\end{align*}
}

\only<3>{
Then
\begin{align*}
\mathbb{E}\left(u\right) &= \mathbf{0}_{J\times 1}\\
\mathbb{V}\left(u\right) &= \Lambda\Sigma_{\boldsymbol{\theta}}\Lambda' + \Omega\\
\Sigma_{\boldsymbol{\theta}} &= \left[\begin{array}{cc}
\sigma^2_{\theta_1} & \sigma_{\theta_1 \theta_2}\\
\sigma_{\theta_1 \theta_2} & \sigma^2_{\theta_2}\\
\end{array}\right]
\end{align*}
}
\end{frame}




\begin{frame}

Our only data source to estimate $\Lambda$ and $\Sigma_{\boldsymbol{\theta}}$ is $\mathbb{V}\left(M-X\boldsymbol{\beta}\right)\equiv \mathbb{V}\left(u\right)$

\bigskip{}

\onslide<2->{
Let's look at the variance-covariance matrix of $u$:

\bigskip{}

\begin{itemize}
\itemsep1.5em
\item<3-> This has $J$ diagonal elements and $\frac{J(J-1)}{2}$ unique off-diagonal elements
\end{itemize}
}

\bigskip{}

\onslide<4->{
With these $J+\frac{J(J-1)}{2}$ moments in the data, we want to estimate:
\bigskip\par
\begin{itemize}
\itemsep1.5em
\item<5-> The $J$ diagonal elements of $\Omega$ (i.e. the $\sigma^2_j$'s)
\item<6-> $2J$ elements of $\Lambda$
\item<7-> three elements of $\Sigma_{\boldsymbol{\theta}}$
\end{itemize}
}

\end{frame}

\begin{frame}

\begin{itemize}
\itemsep1.5em
\item<1-> We have $3J+3$ parameters, but only $J+\frac{J(J-1)}{2}$ data moments
\item<2-> In general, the model is not identified. Need to impose further assumptions
\end{itemize}

\end{frame}

\begin{frame}

The following are common assumptions, but you could impose others

\bigskip{}

\begin{enumerate}
\itemsep1.5em
\item<2-> $\theta_1 \perp \theta_2$ (so $\Sigma_{\boldsymbol{\theta}}$ is diagonal)
\item<3-> The scale of each factor is arbitrary. 2 ways to normalize the scale:
    \bigskip\par
    \begin{itemize}
    \itemsep1.5em
    \item<4-> $\Sigma_{\boldsymbol{\theta}} = I_{2\times 2}$
    \medskip\par or \vspace{-0.25cm}
    \item<5-> Set one element of each row of $\Lambda=1$
    \end{itemize}
\end{enumerate}
\bigskip\par
\onslide<6->{
With these two assumptions, we achieve identification if
\begin{align*}
2J + J \leq J+\frac{J(J-1)}{2}
\end{align*}
}
\onslide<7->{
So $J\geq 5$ is necessary (but not sufficient) for identification
}

\end{frame}

\begin{frame}

\begin{itemize}
\itemsep1.5em
\item<1-> For \textcolor{navy}{model interpretability}, also need to put more structure on $\Lambda$
\item<2-> For example, suppose we have 6 measurements: 
\bigskip\par
    \begin{itemize}
    \itemsep1.5em
    \item<3-> 3 from a cognitive test and 3 from a personality test
    \end{itemize}
\item<4-> In this case, the first row of $\Lambda$ should be 0 for the personality measures
\item<5-> Likewise, the second row of $\Lambda$ should be 0 for the cognitive measures
\item<6-> If all 6 measurements come from a cog. test, can't identify a non-cog. factor
\item<7-> Could possibly identify $\sigma_{\theta_1\theta_2}$ if overlap in measurements
\end{itemize}

\end{frame}

\end{document}