\documentclass[aspectratio=169]{beamer}

\usetheme{default}
\setbeamertemplate{navigation symbols}{}
\setbeamertemplate{itemize item}{\color{black}\textbullet}
\setbeamertemplate{itemize subitem}{\color{black}\textbullet}
\usepackage{xcolor}
\definecolor{navy}{RGB}{0, 0, 128}

\begin{document}

\begin{frame}

The infinite horizon value function:

\bigskip{}

\begin{align*}
v_{j}(X_{i})&=u_{j}(X_{i})+\beta \int_{X_i'}V(X_i')dF_{j}(X_i'|X_{i})
\end{align*}

\bigskip{}

\onslide<2->{
Expanding the continuation value:

\begin{align*}
v_{j}(X_{i})&=u_{j}(X_{i})+\beta \int_{X_i'}E_{\epsilon'}\left(\max_k v_{k}(X_i')+\epsilon'_{ik}\right)dF_{j}(X_i'|X_{i})
\end{align*}
}

\end{frame}

\begin{frame}

For T1EV $\epsilon$'s, the expectation of the maximum has a closed form:

\begin{align*}
v_j(X_i)=u_j(X_i)+\beta\int_{X_i'}\log\left(\sum_{k=1}^J\exp[v_{k}(X_i')]\right)dF_j(X_i'|X_i)+\beta c
\end{align*}

\bigskip{}

\onslide<2->{
Key insight: The $v$'s appear on both sides of the equation

\bigskip{}

This creates a system that requires solving for a fixed point
}

\onslide<3->{
\bigskip{}

(This works because it is a contraction mapping)
}

\end{frame}

\begin{frame}

Let $\mathcal{X}$ denote the number of states $X$ can take on

\bigskip{}

Stack the conditional value functions for each possible state and choice:

\bigskip{}

\only<1>{
\small
\begin{align*}
\left[\begin{array}{c}v_1(X_1)\\ v_1(X_2)\\ \vdots \\ v_1(X_{\mathcal{X}})\\ \vdots\\ v_{J}(X_{\mathcal{X}})\end{array}\right]&=
\left[\begin{array}{c}u_1(X_1)+\beta\int_{X'}\log\left(\sum_{k=1}^J\exp[v_{k}(X')]\right)dF_1(X'|X_1)+\beta c\\ u_1(X_2)+\beta\int_{X'}\log\left(\sum_{k=1}^J\exp[v_{k}(X')]\right)dF_1(X'|X_2)+\beta c\\ \vdots\quad\quad\quad\quad\quad\quad\quad\quad\quad\quad\vdots\quad\quad\quad\quad\quad\quad\quad\quad\quad\quad\vdots\\ u_1(X_{\mathcal{X}})+\beta\int_{X'}\log\left(\sum_{k=1}^J\exp[v_{k}(X')]\right)dF_1(X'|X_{\mathcal{X}})+\beta c\\ \vdots\quad\quad\quad\quad\quad\quad\quad\quad\quad\quad\vdots\quad\quad\quad\quad\quad\quad\quad\quad\quad\quad\vdots\\ u_J(X_{\mathcal{X}})+\beta\int_{X'}\log\left(\sum_{k=1}^J\exp[v_{k}(X')]\right)dF_J(X'|X_{\mathcal{X}})+\beta c\end{array}\right]
\end{align*}
}

\onslide<2->{
This system of equations can be written compactly as:
\begin{align*}
\mathbf{v} = \mathbf{u} + \beta \mathbf{T}(\mathbf{v})
\end{align*}

where $\mathbf{T}(\cdot)$ represents the expectation operator
}

\end{frame}

\begin{frame}

Solution algorithm:

\bigskip{}

\begin{itemize}
\item Plug in values for the parameters and take a guess at the $v$'s
\end{itemize}

\onslide<2->{
\begin{itemize}
\item Substitute in for the $v$'s on the right hand side which gives us a new set of $v$'s
\end{itemize}
}

\onslide<3->{
\begin{itemize}
\item Repeat until convergence
\end{itemize}
}

\bigskip{}

\onslide<4->{
Formally: 
\bigskip{}

Iterate 
\begin{align*}
    \mathbf{v}^{(n+1)} = \mathbf{u} + \beta \mathbf{T}(\mathbf{v}^{(n)})
\end{align*}
until 
\begin{align*}
\|\mathbf{v}^{(n+1)} - \mathbf{v}^{(n)}\| < \delta
\end{align*}
}

\end{frame}


\begin{frame}

The complete estimation algorithm has a similar nested structure as finite horizon case:

\bigskip{}

\textcolor{navy}{Outer loop (parameter estimation):}
\begin{itemize}
\item Guess parameter values $\boldsymbol{\theta}^{(0)}$ (utility coefficients, $\beta$, etc.)
\end{itemize}
\bigskip{}

\onslide<2->{
\textcolor{navy}{Inner loop (value function solution):}
\begin{itemize}
\item For given $\boldsymbol{\theta}^{(m)}$, solve fixed point: $\mathbf{v}^{(n+1)} = \mathbf{u}(\boldsymbol{\theta}^{(m)}) + \beta \mathbf{T}(\mathbf{v}^{(n)})$
\item Continue until $\|\mathbf{v}^{(n+1)} - \mathbf{v}^{(n)}\| < \delta$
\end{itemize}
\bigskip{}

}

\onslide<3->{
\textcolor{navy}{Back to outer loop:}
\begin{itemize}
\item Calculate choice probabilities using solved $\mathbf{v}(\boldsymbol{\theta}^{(m)})$
\item Evaluate likelihood $\mathcal{L}(\boldsymbol{\theta}^{(m)})$
\item Update parameters to $\boldsymbol{\theta}^{(m+1)}$ using nonlinear optimizer (e.g. L-BFGS)
\item Repeat until likelihood convergence
\end{itemize}
\bigskip{}

}

\end{frame}

\end{document}