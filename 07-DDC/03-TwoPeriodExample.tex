\documentclass[aspectratio=169]{beamer}

\usetheme{default}
\setbeamertemplate{navigation symbols}{}
\setbeamertemplate{itemize item}{\color{black}\textbullet}
\setbeamertemplate{itemize subitem}{\color{black}\textbullet}
\setbeamertemplate{enumerate item}{\color{navy}\arabic{enumi}.}
\usepackage{xcolor}
\definecolor{navy}{RGB}{0, 0, 128}
\definecolor{lightblue}{RGB}{230,240,250}
\newcommand{\highlight}[1]{\colorbox{lightblue}{$\displaystyle\textcolor{navy}{#1}$}}
\newcommand{\highlighttext}[1]{\colorbox{lightblue}{\textcolor{navy}{#1}}}

\begin{document}


\begin{frame}
\onslide<1->{
Final period utility for choice $j$ in choice set $\mathcal{J}$:
\begin{align*}
U_{ijT} &= u_{ijT} + \epsilon_{ijT}\\
&\\
&= X_{iT}\alpha_j + \epsilon_{ijT}
\end{align*}

\bigskip{}
}

\onslide<2->{
\textcolor{navy}{Conditional value function} for choice $j$: 
\bigskip{}

\only<2>{deterministic utility today plus discounted expected future utility given today's choice}
\only<3>{\highlighttext{deterministic utility today} plus discounted expected future utility given today's choice}
\only<4>{deterministic utility today plus \highlighttext{discounted expected future utility given today's choice}}

\only<2>{
\begin{align*}
v_{ijT-1} &= u_{ijT-1} + \beta \mathbb{E}\max_{k\in \mathcal{J}}\left\{u_{ikT}+\epsilon_{ikT}|d_{iT-1}=j\right\}
\end{align*}
}

\only<3>{
\begin{align*}
v_{ijT-1} &= \highlight{u_{ijT-1}} + \beta \mathbb{E}\max_{k\in \mathcal{J}}\left\{u_{ikT}+\epsilon_{ikT}|d_{iT-1}=j\right\}
\end{align*}
}

\only<4>{
\begin{align*}
v_{ijT-1} &= u_{ijT-1} + \highlight{\beta \mathbb{E}\max_{k\in \mathcal{J}}\left\{u_{ikT}+\epsilon_{ikT}|d_{iT-1}=j\right\}}
\end{align*}
}

\smallskip{}

where $\beta$ is the discount factor
}
\end{frame}



\begin{frame}
\textcolor{navy}{Bellman Equation} for dynamic discrete choice:

\begin{align*}
V_{it} &= \mathbb{E}\max_{j \in \mathcal{J}}\left\{u_{ijt} + \beta V_{it+1}(X_{it+1}) + \epsilon_{ijt}\right\}
\end{align*}

\bigskip{}

\onslide<2->{
As with $v_{ijt}$ formulation, we have immediate and discounted future components
\bigskip{}
}

\onslide<3->{
$V$ is also called the \textcolor{navy}{unconditional value function}
\bigskip{}
}

\onslide<4->{
Expectation taken over both future $\epsilon$'s and future $X$'s
\bigskip{}
}

\onslide<5->{
Recursive formulation helps keep notation compact
\bigskip{}
}

\end{frame}




\begin{frame}
\onslide<1->{
We've seen the $\mathbb{E}\max$ term before:
\bigskip{}

\begin{itemize}
\itemsep1.5em
    \item When talking about expected consumer surplus
    \item When thinking about nested logit as a multi-stage problem
\end{itemize}

\bigskip{}
}

\onslide<2->{
What is $\mathbb{E}\max$ if $X_{iT}$ is deterministic given $X_{iT-1}$ and $d_{iT-1}$, and $\epsilon$'s are T1EV?
}

\onslide<3->{
\begin{align*}
\mathbb{E}\max = \log\left(\sum_k \exp\left(u_{ikT}\right)\right) + \underbrace{c}_{\text{Euler's\,\,constant}}
\end{align*}
\bigskip{}

}

\end{frame}



\begin{frame}
We can estimate $\alpha$'s by MLE under the following conditions:
\bigskip{}

\begin{itemize}
\itemsep1.5em
    \item $X$'s transition deterministically
    \item $\displaystyle \mathbb{E}\max$ term has a closed form
\end{itemize}
\bigskip{}

\onslide<2->{
Then
\begin{align*}
    v_{ijT} &= X_{iT}\alpha_j,\\
    &\\
    v_{ijT-1}&=X_{iT-1}\alpha_j + \beta \log\sum_{k\in\mathcal{J}}\exp\left(X_{iT}\alpha_k\right) + \beta c
\end{align*}
}

\onslide<3->{
\bigskip{}

Treat the $v$'s like we would the $u$'s in a multinomial logit model
}
    
\end{frame}




\begin{frame}

When might dynamics not matter? \only<1>{Let's walk through location normalization of utility}

\only<2,4>{
\begin{align*}
v_{ijT-1}-v_{ij'T-1} &= u_{ijT-1} - u_{ij'T-1}+\\
&\phantom{==} \beta \mathbb{E}\max_{k}\left\{u_{ikT}+\epsilon_{ikT}|d_{iT-1}=j\right\}-\\
&\phantom{==} \beta \mathbb{E}\max_{k}\left\{u_{ikT}+\epsilon_{ikT}|d_{iT-1}=j'\right\}
\end{align*}
}

\only<3>{
\begin{align*}
v_{ijT-1}-v_{ij'T-1} &= u_{ijT-1} - u_{ij'T-1}+\\
&\phantom{==} \highlight{\beta \mathbb{E}\max_{k}\left\{u_{ikT}+\epsilon_{ikT}|d_{iT-1}=j\right\}}-\\
&\phantom{==} \highlight{\beta \mathbb{E}\max_{k}\left\{u_{ikT}+\epsilon_{ikT}|d_{iT-1}=j'\right\}}
\end{align*}
}



\onslide<3->{
If future value terms are equal $\Rightarrow$ cancellation (model becomes static)

\bigskip{}

Thus, dynamics require that choices affect future states
}

\onslide<4->{
\bigskip{}

Easiest way to satisfy this is switching costs

\bigskip{}

Intuition: switching costs make agents consider future consequences
}

\end{frame}






\end{document}