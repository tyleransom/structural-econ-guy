\documentclass[aspectratio=169]{beamer}

\usetheme{default}
\setbeamertemplate{navigation symbols}{}
\setbeamertemplate{enumerate item}{\color{navy}\arabic{enumi}.}
\setbeamertemplate{itemize item}{\color{black}\textbullet}
\setbeamertemplate{itemize subitem}{\color{black}\textbullet}
\usepackage{booktabs}
\usepackage{xcolor}
\usepackage{tikz}
\usetikzlibrary{shapes,arrows,positioning}
\definecolor{navy}{RGB}{0, 0, 128}
\definecolor{lightblue}{RGB}{230,240,250}
\definecolor{darkgreen}{RGB}{0,100,0}
\definecolor{lightgreen}{RGB}{230,250,230}
\newcommand{\highlight}[1]{\colorbox{lightblue}{$\displaystyle\textcolor{navy}{#1}$}}
\newcommand{\highlighttext}[1]{\colorbox{lightblue}{\textcolor{navy}{#1}}}
\newcommand{\highlightgreen}[1]{\colorbox{lightgreen}{$\displaystyle\textcolor{darkgreen}{#1}$}}

\usepackage{hyperref}
\hypersetup{
    colorlinks=true,
    linkcolor=navy,
    urlcolor=navy,
    citecolor=navy
}

\begin{document}

\begin{frame}
What would Bayesian updating look like if $a_i$ were a vector rather than a scalar?
\bigskip\par
\begin{itemize}
\itemsep1.5em
\item<2-> Let $A_i$ denote the vector, and suppose its population covariance is $\Delta$
\item<3-> $\mathbf{S}_{it} = A_i + \boldsymbol\varepsilon_{it}$ is a vector-valued signal
\end{itemize}
\onslide<4->{
\begin{align*}
\mathbb{E}_{t+1}[A_i]&=(\mathbb{V}^{-1}_{t}[A_i] + \Omega_{it})^{-1}(\mathbb{V}^{-1}_{t}[A_i]\mathbb{E}_{t}[A_i]+\Omega_{it}\mathbf{S}_{it}) \\[1em]
\mathbb{V}_{t+1}[A_i]&=(\mathbb{V}^{-1}_{t}[A_i] + \Omega_{it})^{-1}
\end{align*}
}
\begin{itemize}
\itemsep1.5em
\item<5-> $\Omega_{it}$ is a diagonal matrix with $\frac{1}{\sigma^2_{\varepsilon_j}}$ in the $(j,j)$ element
\item<6-> Elements of $\Omega$ and $\mathbf{S}$ are set to 0 for signals that aren't received
\bigskip\par
\begin{itemize}
\itemsep1.5em
\item<7-> (i.e., not all signals need to be received in every period)
\end{itemize}
\end{itemize}

\end{frame}

\begin{frame}
Let's go through a scalar example:
\bigskip\par
\begin{itemize}
\itemsep1.5em
\item<2-> In period 1, the individual begins with prior beliefs $(\mathbb{E}_1[a_i],\mathbb{V}_1[a_i])$
\item<3-> Usually, set these to the population values $(0,\sigma^2_a)$ for all individuals
\item<4-> Then, a signal $S_{i1}$ is received and beliefs are updated according to the formulas:
\end{itemize}

\onslide<5->{
\begin{align*}
\mathbb{E}_{2}[a_i] &= \underbrace{\mathbb{E}_1[a_i]}_{0}\frac{\sigma^2_\varepsilon}{\sigma^2_\varepsilon + \underbrace{\mathbb{V}_1[a_i]}_{\sigma^2_a}} + S_{i1}\frac{\overbrace{\mathbb{V}_1[a_i]}^{\sigma^2_a}}{\sigma^2_\varepsilon + \underbrace{\mathbb{V}_1[a_i]}_{\sigma^2_a}}\\
& = \frac{S_{i1}\sigma^2_a}{\sigma^2_\varepsilon + \sigma^2_a}
\end{align*}
}

\end{frame}

\begin{frame}

When $S_{i1}$ is received, $i$ updates the variance as follows:

\bigskip{}

\onslide<2->{
\begin{align*}
\mathbb{V}_{2}[a_i] &= \underbrace{\mathbb{V}_1[a_i]}_{\sigma^2_a} \frac{\sigma^2_\varepsilon}{\sigma^2_\varepsilon + \underbrace{\mathbb{V}_1[a_i]}_{\sigma^2_a}}\\
& = \frac{\sigma^2_a\sigma^2_\varepsilon}{\sigma^2_\varepsilon + \sigma^2_a}
\end{align*}
}

\bigskip{}

\onslide<3->{
It is straightforward to show that $\mathbb{V}_{2}[a_i]<\mathbb{V}_{1}[a_i]$ when $\sigma^2_a>0$
}

\end{frame}


\begin{frame}
Back to the vector example:
\bigskip\par
\begin{itemize}
\itemsep1.5em
\item<2-> Division in scalar form is replaced with matrix inversion
\item<3-> Multiplication is replaced with matrix multiplication
\item<4-> Other matrices (e.g. $\Omega$) defined for conformability
\end{itemize}

\end{frame}



\end{document}