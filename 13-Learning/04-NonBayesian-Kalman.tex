\documentclass[aspectratio=169]{beamer}

\usetheme{default}
\setbeamertemplate{navigation symbols}{}
\setbeamertemplate{enumerate item}{\color{navy}\arabic{enumi}.}
\setbeamertemplate{itemize item}{\color{black}\textbullet}
\setbeamertemplate{itemize subitem}{\color{black}\textbullet}
\usepackage{booktabs}
\usepackage{xcolor}
\usepackage{tikz}
\usetikzlibrary{shapes,arrows,positioning}
\definecolor{navy}{RGB}{0, 0, 128}
\definecolor{lightblue}{RGB}{230,240,250}
\definecolor{darkgreen}{RGB}{0,100,0}
\definecolor{lightgreen}{RGB}{230,250,230}
\newcommand{\highlight}[1]{\colorbox{lightblue}{$\displaystyle\textcolor{navy}{#1}$}}
\newcommand{\highlighttext}[1]{\colorbox{lightblue}{\textcolor{navy}{#1}}}
\newcommand{\highlightgreen}[1]{\colorbox{lightgreen}{$\displaystyle\textcolor{darkgreen}{#1}$}}

\usepackage{hyperref}
\hypersetup{
    colorlinks=true,
    linkcolor=navy,
    urlcolor=navy,
    citecolor=navy
}

\begin{document}

\begin{frame}

\begin{itemize}
\itemsep1.5em
\item<1-> Bayesian updating is so popular because the math works out nicely

\item<2-> This is because of what is known as a \textcolor{navy}{conjugate prior} (read Wikipedia)

\item<3-> If we assume a different kind of updating, the math could get ugly real quick

\item<4-> But Bayesian updating has intuitive properties (e.g. $\lim_{t\rightarrow\infty}\mathbb{V}_{t+1}[a_i] = 0$)

\item<5-> Moreover, we often don't know people's beliefs

\item<6-> If we had detailed data on people's beliefs, that would allow us to be more flexible
\bigskip\par
\begin{itemize}
\itemsep1.5em
\item cf. stated probabilistic choice models and quantification of preferences
\end{itemize}
\end{itemize}

\end{frame}

\begin{frame}

\begin{itemize}
\itemsep1.5em
\item<1-> Things get more complicated if the signal is not continuous
\bigskip\par
\begin{itemize}
\itemsep1.5em
\item<2-> Naturally, a discrete signal will provide less information

\item<3-> e.g. Pass/Fail on an exam, versus a 0-100 score
\end{itemize}

\item<4-> Another complication is if the signal is selected
\bigskip\par
\begin{itemize}
\itemsep1.5em
\item<5-> For example, I only see a wage signal if I have a job

\item<6-> In this case, we need a choice model to resolve the sample selection problem

\end{itemize}
\item<7-> Or if the signal is censored
\bigskip\par
\begin{itemize}
\itemsep1.5em
\item<8-> e.g. Grade Point Average capped at ``perfect grades''

\end{itemize}
\end{itemize}

\end{frame}

\begin{frame}

\begin{itemize}
\itemsep1.5em
\item<1-> The \textcolor{navy}{Kalman filter} is a special type of Bayesian updating of a learning model

\item<2-> Most common application: remote sensing of aircraft/spacecraft
\bigskip\par
\begin{itemize}
\itemsep1.5em
\item<3-> Any given sensor sends back a ``noisy'' signal about exact location

\item<4-> Multiple sensors acting in sequence can provide more reliable location info
\end{itemize}


\item<5-> Other applications: player/team skill ratings
\bigskip\par
\begin{itemize}
\itemsep1.5em
\item<6-> In chess/video games, ``Glicko'' system generalizes ELO to allow for uncertainty

\end{itemize}
\end{itemize}

\end{frame}

\end{document}