\documentclass[aspectratio=169]{beamer}

\usetheme{default}
\setbeamertemplate{navigation symbols}{}
\setbeamertemplate{itemize item}{\color{black}\textbullet}
\setbeamertemplate{itemize subitem}{\color{black}\textbullet}
\usepackage{xcolor}
\definecolor{navy}{RGB}{0, 0, 128}

\begin{document}

\begin{frame}

One solution to the homogeneity problem is to add interaction terms

\bigskip{}

Suppose we have a 2-option commute model, where $X_i$ is student status and $Z$ is price:

\only<1>{
\bigskip{}
\bigskip{}

\begin{align*}
u_{i,bus}&=\beta_1 X_i + \gamma Z_1\\
u_{i,car}&=\beta_2 X_i + \gamma Z_2
\end{align*}
}

\onslide<2->{
\bigskip{}

We could introduce heterogeneity in $\gamma$ by interacting $Z_j$ with $X_i$:

\begin{align*}
u_{i,bus}&=\beta_1 X_i + \widetilde{\gamma} Z_1 X_i\\
u_{i,car}&=\beta_2 X_i + \widetilde{\gamma} Z_2 X_i
\end{align*}
}

\onslide<3->{
Now a change in $Z_j$ will have a heterogeneous impact on utility depending on $X_i$

\bigskip{}

e.g. students $(X_i=1)$ may be more/less sensitive to changes in price $(Z_j)$
}

\end{frame}

\begin{frame}

Observable preference heterogeneity can be useful

\bigskip{}

\onslide<2->{
But many dimensions of preferences are likely unobserved

\bigskip{}

In this case, we need to ``interact'' $Z$ with something unobserved
}

\bigskip{}

\onslide<3->{
One way to do this is to assume that $\beta$ or $\gamma$ varies across people

\bigskip{}

Assume some distribution for $\beta$ or $\gamma$ (e.g. Normal), called the \textcolor{navy}{mixing distribution}
}

\bigskip{}

\onslide<4->{
Then integrate this out of the likelihood function
}

\bigskip{}

\onslide<5->{
Coefficients that are ``mixed'' are called \textcolor{navy}{random coefficients}
}

\bigskip{}

\onslide<5->{
Models with discrete mixing distributions: \textcolor{navy}{latent class models} or \textcolor{navy}{finite mixture models}
}

\end{frame}

\end{document}