\documentclass[aspectratio=169]{beamer}

\usetheme{default}
\setbeamertemplate{navigation symbols}{}
\setbeamertemplate{enumerate item}{\color{navy}\arabic{enumi}.}
\setbeamertemplate{itemize item}{\color{black}\textbullet}
\setbeamertemplate{itemize subitem}{\color{black}\textbullet}
\usepackage{booktabs}
\usepackage{xcolor}
\usepackage{tikz}
\usetikzlibrary{shapes,arrows,positioning}
\definecolor{navy}{RGB}{0, 0, 128}
\definecolor{lightblue}{RGB}{230,240,250}
\definecolor{darkgreen}{RGB}{0,100,0}
\definecolor{lightgreen}{RGB}{230,250,230}
\newcommand{\highlight}[1]{\colorbox{lightblue}{$\displaystyle\textcolor{navy}{#1}$}}
\newcommand{\highlighttext}[1]{\colorbox{lightblue}{\textcolor{navy}{#1}}}
\newcommand{\highlightgreen}[1]{\colorbox{lightgreen}{$\displaystyle\textcolor{darkgreen}{#1}$}}

\begin{document}

\begin{frame}

Model validation addresses skepticism about model fit

\bigskip{}

\begin{itemize}
\itemsep1.5em
\item<2-> Using entire dataset for estimation risks \textcolor{navy}{overfitting}
\item<3-> Matching targeted moments can also overfit
\item<4-> \textcolor{navy}{Model validation} demonstrates model generalizes beyond data used for estimation
\item<5-> i.e. for counterfactual prediction
\item<6-> A valid model:
\begin{itemize}
    \item<7->[]
    \item<7-> Reproduces patterns in a ``holdout sample'' excluded from estimation
    \item<7->[]
    \item<8-> Or matches moments of estimation sample not explicitly targeted
\end{itemize}

\end{itemize}

\end{frame}

\begin{frame}

Validation assesses whether the model ``makes sense''

\bigskip{}

\begin{itemize}
\itemsep1.5em
\item<2-> Qualitative: Do parameters conform to theory?
\item<3-> Quantitative: Is prediction error sufficiently low?
\item<4-> Machine learning uses \textcolor{navy}{cross-validation} to determine optimal complexity
\item<5-> Cross-validation enables automated specification search
\item<6-> Structural estimation rarely uses automated specification search
\end{itemize}

\end{frame}

\begin{frame}

\textcolor{navy}{Lang \& Palacios (2018, NBER WP)}

\bigskip{}

Key question: How much unobserved heterogeneity should we specify?

\bigskip{}

\begin{itemize}
\itemsep1.5em
\item<2-> Cannot determine optimal amount \textit{ex ante}
\item<3-> Typical approach: Add ``enough'' for good in-sample fit
\item<4-> Authors search for optimal number of types
\item<5-> Use 20\% holdout sample for out-of-sample assessment
\item<6-> Formally compare model fit between estimation and holdout data
\item<7-> Take-away: Visual model fit often insufficient
\end{itemize}

\end{frame}

\begin{frame}

\textcolor{navy}{Delavande \& Zafar (2019, JPE)}

\bigskip{}

Alternative validation: Predict treatment effects in control group

\bigskip{}

\begin{itemize}
\itemsep1.5em
\item<2-> Compare model predictions across states of the world
\item<3-> New state: Schooling without costs
\item<4-> Elicited preferences under both cost regimes
\item<5-> Estimated model using only one regime
\item<6-> Tests whether model predicts behavior in counterfactual regime
\end{itemize}

\end{frame}

\begin{frame}

\textcolor{navy}{Arcidiacono et al. (2025, NBER WP)}

\bigskip{}

\begin{itemize}
\itemsep1.5em
\item<2-> Analyze school choice vouchers in India
\item<3-> Use a field experiment to validate structural model
\item<4-> Estimate model on control villages, predict behavior in treatment villages
\item<5-> Initial model: underpredicts take-up, overpredicts school quality
\item<6-> Add new mechanisms based on prediction failures
\item<7-> Enriched model predicts treatment outcomes much better
\item<8-> Final policy experiments use enriched model
\end{itemize}

\end{frame}


\end{document}