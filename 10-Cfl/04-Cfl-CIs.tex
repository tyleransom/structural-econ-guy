\documentclass[aspectratio=169]{beamer}

\usetheme{default}
\setbeamertemplate{navigation symbols}{}
\setbeamertemplate{enumerate item}{\color{navy}\arabic{enumi}.}
\setbeamertemplate{itemize item}{\color{black}\textbullet}
\setbeamertemplate{itemize subitem}{\color{black}\textbullet}
\usepackage{booktabs}
\usepackage{xcolor}
\usepackage{tikz}
\usetikzlibrary{shapes,arrows,positioning}
\definecolor{navy}{RGB}{0, 0, 128}
\definecolor{lightblue}{RGB}{230,240,250}
\definecolor{darkgreen}{RGB}{0,100,0}
\definecolor{lightgreen}{RGB}{230,250,230}
\newcommand{\highlight}[1]{\colorbox{lightblue}{$\displaystyle\textcolor{navy}{#1}$}}
\newcommand{\highlighttext}[1]{\colorbox{lightblue}{\textcolor{navy}{#1}}}
\newcommand{\highlightgreen}[1]{\colorbox{lightgreen}{$\displaystyle\textcolor{darkgreen}{#1}$}}

\begin{document}

\begin{frame}

Confidence intervals around counterfactuals

\bigskip{}

\begin{itemize}
\itemsep1.5em
\item<2-> The counterfactuals produce, e.g., a new predicted distribution of $Y$
\item<3-> But we don't know what the confidence intervals are around the prediction
\item<4-> Our estimates inherently come with sampling variation
\item<5-> We should incorporate that variation into our counterfactual
\item<6-> How do we do this? Bootstrapping
\end{itemize}

\end{frame}

\begin{frame}

Bootstrapping to get CIs around counterfactuals

\bigskip{}

\onslide<2->{
Recall: $-H^{-1}$ is the variance matrix of our estimates, where $H$ is the Hessian
}

\bigskip{}

\onslide<3->{
Assume that $\hat{\theta} \sim MVN(\theta,-H^{-1})$
\bigskip\par
}

\onslide<4->{
\begin{itemize}
\itemsep1.5em
\item but remember that we want to minimize $-\ell$ so we just use $H^{-1}$
\end{itemize}
}

\end{frame}

\begin{frame}

\begin{itemize}
\itemsep1.5em
\item<1-> We sample $B$ times from $MVN(\theta,-H^{-1})$ and compute our new counterfactuals
\item<2-> Then we look at the 2.5th and 97.5th percentiles of cfl statistics
\item<3-> This will represent a 95\% confidence interval of that statistic
\item<4-> This algorithm is known as the \textcolor{navy}{parametric bootstrap}
\item<5-> Contrast with \textcolor{navy}{nonparametric bootstrap} which is when we randomly re-sample data
\item<6-> P.B. can be intensive if it is costly to conduct counterfactuals $\to$ rarely used
\end{itemize}

\end{frame}

\end{document}