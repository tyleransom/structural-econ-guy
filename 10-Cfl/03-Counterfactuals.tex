\documentclass[aspectratio=169]{beamer}

\usetheme{default}
\setbeamertemplate{navigation symbols}{}
\setbeamertemplate{enumerate item}{\color{navy}\arabic{enumi}.}
\setbeamertemplate{itemize item}{\color{black}\textbullet}
\setbeamertemplate{itemize subitem}{\color{black}\textbullet}
\usepackage{booktabs}
\usepackage{xcolor}
\usepackage{tikz}
\usetikzlibrary{shapes,arrows,positioning}
\definecolor{navy}{RGB}{0, 0, 128}
\definecolor{lightblue}{RGB}{230,240,250}
\definecolor{darkgreen}{RGB}{0,100,0}
\definecolor{lightgreen}{RGB}{230,250,230}
\newcommand{\highlight}[1]{\colorbox{lightblue}{$\displaystyle\textcolor{navy}{#1}$}}
\newcommand{\highlighttext}[1]{\colorbox{lightblue}{\textcolor{navy}{#1}}}
\newcommand{\highlightgreen}[1]{\colorbox{lightgreen}{$\displaystyle\textcolor{darkgreen}{#1}$}}

\usepackage{hyperref}
\hypersetup{
    colorlinks=true,
    linkcolor=navy,
    urlcolor=navy,
    citecolor=navy
}


\begin{document}

\begin{frame}

Counterfactuals: change the model, simulate new data, compare with status quo

\bigskip{}

\begin{itemize}
\itemsep1.5em
\item<2-> ``Changing a policy'' typically captured by changing parameter values
\item<3-> ``Policy experiments'' mimic real-world policy this way
\item<4-> \textcolor{navy}{Note:} policy experiments assume invariance of other model parameters
\item<5-> If one parameter changes, it won't ``leak'' into others
\item<6-> This may be a heroic assumption, depending on the model
\end{itemize}

\end{frame}

\begin{frame}

Examples of counterfactuals:

\bigskip{}

\begin{itemize}
\itemsep1.5em
\item<2-> \href{https://www.nber.org/papers/w27838}{Caucutt et al. (2020)}:  How do childcare/goods subsidies affect all parental investments when inputs are complements?
\item<3-> \href{https://onlinelibrary.wiley.com/doi/full/10.3982/QE1722}{Fu et al. (2022)}: Would low-SES youth commit fewer crimes if sent to better schools?
\item<4-> \href{https://jhr.uwpress.org/content/early/2021/03/02/jhr.monopsony.0219-10013R2}{Ransom (2022)}: Would American men move to a new city if paid \$10,000?
\item<5-> \href{https://jhr.uwpress.org/content/early/2022/01/04/jhr.0421-11641R1}{Arcidiacono et al. (2024)}: Would Harvard become less white without legacy admissions?
\item<6-> \href{https://www.journals.uchicago.edu/doi/abs/10.1086/732526}{Arcidiacono et al. (2025)}: How would human capital investment change with perfect information about one's abilities?
\end{itemize}

\end{frame}


\begin{frame}

To perform counterfactual simulations:

\bigskip{}

\begin{itemize}
\itemsep1.5em
\item<2-> Impose parameter restrictions
\item<3-> Simulate fake data under those restrictions
\item<4-> Compare resulting model summary to a ``baseline'' prediction (i.e. model fit)
\end{itemize}

\bigskip{}

\onslide<6->{
If you use SMM, you already have the code to simulate the model
}

\bigskip{}

\onslide<7->{
Otherwise, you'll need to write the code to compute the counterfactuals
}

\end{frame}

\end{document}