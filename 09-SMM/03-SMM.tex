\documentclass[aspectratio=169]{beamer}

\usetheme{default}
\setbeamertemplate{navigation symbols}{}
\setbeamertemplate{enumerate item}{\color{navy}\arabic{enumi}.}
\setbeamertemplate{itemize item}{\color{black}\textbullet}
\setbeamertemplate{itemize subitem}{\color{black}\textbullet}
\usepackage{booktabs}
\usepackage{xcolor}
\usepackage{tikz}
\usetikzlibrary{shapes,arrows,positioning}
\definecolor{navy}{RGB}{0, 0, 128}
\definecolor{lightblue}{RGB}{230,240,250}
\definecolor{darkgreen}{RGB}{0,100,0}
\definecolor{lightgreen}{RGB}{230,250,230}
\newcommand{\highlight}[1]{\colorbox{lightblue}{$\displaystyle\textcolor{navy}{#1}$}}
\newcommand{\highlighttext}[1]{\colorbox{lightblue}{\textcolor{navy}{#1}}}
\newcommand{\highlightgreen}[1]{\colorbox{lightgreen}{$\displaystyle\textcolor{darkgreen}{#1}$}}

\hypersetup{
    colorlinks=true,      % color the text of links instead of boxes
    urlcolor=navy,        % color for \href and \url links
    linkcolor=navy,       % color for internal links (e.g., \ref)
    citecolor=darkgreen   % color for citations, if you use them
}


\begin{document}

\begin{frame}

Simulated Method of Moments

\bigskip

\begin{itemize}
\itemsep1.5em
\item<2-> SMM is a simulated version of GMM
\item<3-> Uses moments from simulated data instead of analytical moments
\item<4-> Objective: make simulated and actual data match
\item<5-> See \href{http://www.jstor.org/stable/1913621}{McFadden (1989)} and \href{https://github.com/QuantEcon/notebook-gallery/blob/main/ipynb/richard_w_evans-smmest.ipynb}{Evans (2018)} for details
\end{itemize}

\end{frame}

\begin{frame}

Pros of SMM

\bigskip

\begin{itemize}
\itemsep1.5em
\item<2-> Can estimate models with $P$'s that don't have closed form (e.g. probit)
\item<3-> Can handle otherwise intractable models (dynamic models, high-dim integrals)
\item<4-> Can estimate micro-models using only aggregated data
\item<5-> Simulation code already done---easy to move to counterfactuals
\item<6-> Straightforward to interpret which moments the model is fitting
\item<7-> Easier to compare with reduced-form evidence
\end{itemize}

\end{frame}

\begin{frame}

Cons of SMM

\bigskip

\begin{itemize}
\itemsep1.5em
\item<2-> Much more computationally intensive than GMM
\item<3-> Must fully specify model (including error distribution)
\item<4-> Loss of statistical efficiency relative to MLE (larger standard errors)
\item<5-> Selection of moments can feel ad hoc
\end{itemize}

\end{frame}

\begin{frame}

SMM Example: Linear Regression

\bigskip

Consider a simple linear regression model:

\begin{align*}
y &= X\beta + \varepsilon,\\
\varepsilon&\sim N(0,\sigma^2)
\end{align*}

\bigskip

\onslide<2->{
$y$ and $X$ are data; we want to estimate $\beta$ and $\sigma$

\bigskip

As mentioned earlier, we must make a strong assumption about DGP: $\varepsilon\sim N(0,\sigma^2)$
}

\end{frame}

\begin{frame}

SMM Estimation Steps

\bigskip

For each guess of $\theta = [\beta', \sigma]'$:

\bigskip

\begin{itemize}
\itemsep1.5em
\item<2-> Compute data moments
\item<3-> Draw $N$ $\varepsilon$'s $D$ times (typically $D>1000$)
\item<4-> For each draw, compute $\tilde{y} = X\beta + \varepsilon$
\item<5-> Compute model moments using $\tilde{y}$ (same structure as data moments)
\item<6-> Average model moments across all $D$ draws
\item<7-> Update objective function: minimize distance between data and model moments
\end{itemize}

\end{frame}




\begin{frame}

Moments to Match

\bigskip

Data moments: $\left\{y_i, i=1,\ldots,N;\widehat{V}(y)\right\}$

\bigskip

\onslide<2->{
Model moments: $\left\{\tilde{y}_i, i=1,\ldots,N;\widehat{V}(\tilde{y})\right\}$
}


\end{frame}

\begin{frame}

SMM Optimization

\bigskip

\begin{itemize}
\itemsep1.5em
\item<2-> Can use any optimizer for the SMM objective function
\item<3-> SMM objective may have local optima (poorly behaved)
\item<4-> Tactics for finding global optimum:
\item<4->[]
    \begin{itemize}
    \itemsep1.5em
    \item Use L-BFGS from many starting values
    \item Use Simulated Annealing or Particle Swarm
    \end{itemize}
\item<5-> Simple problems (like OLS) should be well-behaved
\item<6-> Be sure to use same draw of $\varepsilon$ in every iteration!
\end{itemize}

\end{frame}

\end{document}