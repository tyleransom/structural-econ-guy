\documentclass[aspectratio=169]{beamer}

\usetheme{default}
\setbeamertemplate{navigation symbols}{}
\setbeamertemplate{enumerate item}{\color{navy}\arabic{enumi}.}
\setbeamertemplate{itemize item}{\color{black}\textbullet}
\setbeamertemplate{itemize subitem}{\color{black}\textbullet}
\usepackage{booktabs}
\usepackage{xcolor}
\usepackage{tikz}
\usetikzlibrary{shapes,arrows,positioning}
\definecolor{navy}{RGB}{0, 0, 128}
\definecolor{lightblue}{RGB}{230,240,250}
\definecolor{darkgreen}{RGB}{0,100,0}
\definecolor{lightgreen}{RGB}{230,250,230}
\newcommand{\highlight}[1]{\colorbox{lightblue}{$\displaystyle\textcolor{navy}{#1}$}}
\newcommand{\highlighttext}[1]{\colorbox{lightblue}{\textcolor{navy}{#1}}}
\newcommand{\highlightgreen}[1]{\colorbox{lightgreen}{$\displaystyle\textcolor{darkgreen}{#1}$}}

\usepackage{hyperref}
\hypersetup{
    colorlinks=true,
    linkcolor=navy,
    urlcolor=navy,
    citecolor=navy
}

\begin{document}

\begin{frame}

Example hypothetical choice scenarios from \href{https://www.sciencedirect.com/science/article/abs/pii/S0304407621000415}{Ko\c{s}ar, Ransom and van der Klaauw (2022)}

\bigskip

\onslide<2->{
What is the percent chance you would choose to live in each of these three locations given their characteristics below? \textit{Assume that the locations are otherwise identical.}
}

\bigskip

\onslide<3->{
\textcolor{navy}{Scenario 1}

\begin{table}[h]
\centering
\begin{tabular}{lcccc}
\toprule
Option & Distance & Family here & Income & Probability \\
\midrule
A (not move) & 0 & No & 30\% lower & \\
B & 1000 miles & Yes & same & \\
C & 1000 miles & No & 30\% higher & \\
\bottomrule
\end{tabular}
\end{table}
}

\end{frame}

\begin{frame}

\textcolor{navy}{Scenario 2}

\begin{table}[h]
\centering
\begin{tabular}{lcccc}
\toprule
Option & Distance & Family here & Income & Probability \\
\midrule
A (not move) & 0 & Yes & 30\% lower & \\
B & 500 miles & Yes & 150\% higher & \\
C & 100 miles & No & 60\% higher & \\
\bottomrule
\end{tabular}
\end{table}

\end{frame}


\begin{frame}

Types of SP choice experiments

\bigskip

\begin{itemize}
\itemsep1.5em
\item<2-> Discrete choice
\medskip\par
    \begin{itemize}
    \itemsep1.5em
    \item Individuals select which of the $J$ options they prefer
    \item i.e. if $J=3$ then the $y$ vector will be $[0,1,0]$, $[1,0,0]$, or $[0,0,1]$
    \end{itemize}
\item<3-> Rank-ordered choice
    \begin{itemize}
    \item[]
    \item Individuals provide their preference ordering of the $J$ options
    \end{itemize}
\item<4-> Probabilistic choice
    \begin{itemize}
    \item[]
    \item Individuals provide choice probabilities of each of the $J$ options (must sum to 1)
    \end{itemize}
\end{itemize}

\bigskip

\onslide<5->{
Each of these settings provides increasing amounts of information
}

\end{frame}

\begin{frame}

General considerations in designing choice experiments

\bigskip

\begin{itemize}
\itemsep1.5em
\item<2-> Keep the number of choices limited (2-4 alternatives most common)
\item<3-> Have the choice setup mimic real-life decisions
\item<4-> Include a status-quo or reference option (despite risk of status-quo bias)
\item<5-> Only vary a small subset of attributes at a time
\item<6-> Instruct respondents that alternatives only differ in explicitly listed attributes
\item<7-> Randomly assign respondents to subsets of choice alternatives/attributes
\end{itemize}

\end{frame}

\begin{frame}

Identification and design

\bigskip

\begin{itemize}
\itemsep1.5em
\item<2-> Need independent variation in each attribute to identify preferences
\item<3-> With three attributes: need four choice scenarios with independent variation to identify linear effects
\item<4-> Additional scenarios required for nonlinear effects or interaction effects
\item<5-> Large literature on optimal design variation across scenarios (D-efficient or D-optimal designs)
\end{itemize}

\end{frame}

\begin{frame}

Why elicit choice probabilities?

\bigskip

\begin{itemize}
\itemsep1.5em
\item<2-> Accommodates remaining uncertainty about unspecified attributes or states of the world (resolvable uncertainty)
\item<3-> Provides respondents opportunity to express such uncertainty
\item<4-> Collects richer information than forcing binary choice
\item<5-> Example: if two alternatives highly preferred to third and individual is close to indifferent between first two, this is better captured in choice probabilities
\item<6-> Whether such resolvable uncertainty exists is an empirical question
\end{itemize}

\end{frame}

\end{document}