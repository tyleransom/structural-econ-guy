\documentclass[aspectratio=169]{beamer}

\usetheme{default}
\setbeamertemplate{navigation symbols}{}
\setbeamertemplate{enumerate item}{\color{navy}\arabic{enumi}.}
\setbeamertemplate{itemize item}{\color{black}\textbullet}
\setbeamertemplate{itemize subitem}{\color{black}\textbullet}
\usepackage{booktabs}
\usepackage{xcolor}
\usepackage{tikz}
\usetikzlibrary{shapes,arrows,positioning}
\definecolor{navy}{RGB}{0, 0, 128}
\definecolor{lightblue}{RGB}{230,240,250}
\definecolor{darkgreen}{RGB}{0,100,0}
\definecolor{lightgreen}{RGB}{230,250,230}
\newcommand{\highlight}[1]{\colorbox{lightblue}{$\displaystyle\textcolor{navy}{#1}$}}
\newcommand{\highlighttext}[1]{\colorbox{lightblue}{\textcolor{navy}{#1}}}
\newcommand{\highlightgreen}[1]{\colorbox{lightgreen}{$\displaystyle\textcolor{darkgreen}{#1}$}}

\usepackage{hyperref}
\hypersetup{
    colorlinks=true,
    linkcolor=navy,
    urlcolor=navy,
    citecolor=navy
}

\begin{document}

\begin{frame}

Example data structure in a discrete choice experiment:

\bigskip{}

\begin{table}
\centering
\begin{tabular}{ccccc}
\toprule
ID & Scenario & Alternative & Chosen & $Z$ \\
\midrule
1 & 1 & A & 1 & 2.3 \\
1 & 1 & B & 0 & 1.8 \\
1 & 1 & C & 0 & 3.1 \\ \midrule
1 & 2 & A & 0 & 2.5 \\
1 & 2 & B & 1 & 2.9 \\
1 & 2 & C & 0 & 1.7 \\ \midrule
2 & 1 & A & 0 & 3.2 \\
2 & 1 & B & 1 & 4.1 \\
2 & 1 & C & 0 & 2.8 \\
\bottomrule
\end{tabular}
\end{table}

\bigskip{}

\onslide<2->{
Each individual faces multiple scenarios, each with multiple alternatives
}

\end{frame}




\begin{frame}

\begin{itemize}
\itemsep1.5em
\item<1-> Estimation proceeds as if one has panel data on choices (Rust, 1987)
\item<2-> Each choice scenario is another observation in the individual's panel
\item<3-> Can estimate assuming multinomial logit, nested logit, mixed logit, etc.
\end{itemize}

\end{frame}


\begin{frame}

Example data structure for rank-ordered logit:

\bigskip{}

\begin{table}
\centering
\begin{tabular}{ccccc}
\toprule
ID & Scenario & Alternative & Rank & $Z$ \\
\midrule
1 & 1 & A & 1 & 2.3 \\
1 & 1 & B & 3 & 1.8 \\
1 & 1 & C & 2 & 3.1 \\ \midrule
1 & 2 & A & 2 & 2.5 \\
1 & 2 & B & 1 & 2.9 \\
1 & 2 & C & 3 & 1.7 \\\midrule
2 & 1 & A & 3 & 3.2 \\
2 & 1 & B & 1 & 4.1 \\
2 & 1 & C & 2 & 2.8 \\
\bottomrule
\end{tabular}
\end{table}

\bigskip{}

\onslide<2->{
Rank $= 1$ indicates most preferred alternative in each scenario
}

\end{frame}




\begin{frame}

Here, we use the \textcolor{navy}{exploded logit} model (Beggs et al., 1981) for estimation

\bigskip{}

\onslide<2->{
The ``choice probability'' is the joint event of a particular ranking of options
}

\bigskip{}

\onslide<3->{
It's a product of logit $P$'s, where the choice set decreases as options are ranked:

\begin{align*}
\Pr\left(\text{Ranking}=1,\ldots,J\right) &= \frac{\exp\left(Z_{i1}\gamma\right)}{\sum_{k=1}^J\exp\left(Z_{ik}\gamma\right)}\frac{\exp\left(Z_{i2}\gamma\right)}{\sum_{k=2}^J\exp\left(Z_{ik}\gamma\right)}\cdots\frac{\exp\left(Z_{iJ-1}\gamma\right)}{\sum_{k=J-1}^J\exp\left(Z_{ik}\gamma\right)}
\end{align*}
}

\end{frame}

\begin{frame}

\begin{itemize}
\itemsep1.5em
\item<1-> We can also add mixing to these probabilities to get a \textcolor{navy}{mixed exploded logit}
\item<2-> Note that rank ordering provides more information than 0/1 choice data
\item<3-> We now know the relative preference of the $J-1$ non-chosen options
\end{itemize}

\end{frame}

\end{document}